%!TEX root = 0-main.tex

% Author: Philipp Moers <soziflip@gmail.com> 




\chapter{Turingmaschinen, Berechenbarkeit und Komplexität} % (fold)
\label{cha:turingmaschinen_berechenbarkeit_und_komplexitaet}


\datum{13.10.15}

\section{Padding}

Sei L \subseteq \sigma * eine Sprache

(Blase: 2^?  ist die PAddingfunktion geht auch fuer andere P.fkt.
padd(L) = {1^l O x | x \in L, l = 2^|x|}


Es gilt L \in DTime (f(s^n)), dann ist
padd(l) \in DTIME(f(n)) ... Blase: f zeitkonstant und insbesondere f(n) >=n

Begruendung:  Sei T eine DTM fuer L und
DTIME_T(x) <= \in xf(2^n)
Die folgende Maschine T' entscheided padd(L) :
Gegeben Eingabe y schreibe y + 1^lOx und pruefe ob l=2^|x|
geht in Zeit \bigO(|y|)

Aufwand c f(2^|x|)

<= c f(|y|)

Gesamtaufwand
<= c f(|y|) + |y|
= \bigO(f(|y|))
falls f(n) >= n (zeitkonstruierbar)


Umgegekrt gilt auch Wenn padd(l) \in DTIME(f(n))
dann L\in DTIME(f(s^{n+1}))
Sei T eine M. fuer padd(L)
mit DTIME_T(y) <= c f(|y|)
Wir bauen eine MAschine fuer L:
Gege. Eingabe x.  Bilde y=1^2^{|x|} O x
Aufwand \bigO(2^{|x|})
Setze T auf y an.
Aufwand: c f (|y|) = c f (2^{|x|} +|x| + 1)

Gesamtaufwand
O(f(2^{|x|+1}
Folgerung
P=NP \Rightarrow E=NE
Begruendung: Sei P = NP
und L \in NE und T eine Maschine mit Aufwand n^{kk}
k fest, n = Laenge der Eingabe.

L \in NTIME(n^{nk}) = NTIME((2^n)^k)
Also padd(L) \in NTIME  (2^k)
also padd(L) \in  NP und nach Annahme
padd(L) \in P

Also padd(L) \in DTIME(n^{k'})
also L \in DTIME((2^{n+1})^{k'})
= DTIME(2^{k'n + k') = DTIME(2^{k'n})
  \subseteq E


  Slogan: Gleichheit von Komplexitaetsklassen vererbt sich nach oben

  Mit anderen PAddingfunktion zeigt man ebenso:
  P = NP \Rightarrow EXP = NEXP
  E = NE \Rightarrow EXP = NEXP

  Kontrapositiv ausgedrueckt.

  E\neq NE \Rightarrow P\neq NP ect.

  Es koennte sein, dass P \neq NP aber doch E=NE

  Slogan: Trennung von Komplexitätklassen vererbt sich durch Padding) von oben nach unten.

  --------------------------------

  Bemerkung zu P

  \begin{enumerate}
    
  \item Wird gemeinhin gleichgesetzet mit ``effizient loesbar''.

  \item Wachstumsverhalten

  p Polynom. \Rightarrow es gibt  c>0 so dass gilt

  P(2n) <= c p (n)

  Input verdoppelt \Rightarrow Output ver c facht.
  \item  Haeufig hat brute force Loesung ``stures Durchprobieren''
  exponentiellen und eine echte algorithmesche Loesung hat polynomiellen Aufwand.



  \end{enumerate}

  


  
\begin{definition}
        

\end{definition}



\begin{beispiel}


\end{beispiel}



% chapter turingmaschinen_berechenbarkeit_und_komplexitaet (end)
