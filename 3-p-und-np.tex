%!TEX root = 0-main.tex

% Author: Philipp Moers <soziflip@gmail.com> 




\chapter{NP und P} % (fold)
\label{cha:np_und_p}
        

\section{Padding}

\begin{definition}
    Sei $L \subseteq \Sigma^\ast$ eine Sprache.

    % (Blase: 2^?  ist die Paddingfunktion geht auch für andere P.fkt.)
    % TODO verstehen, anpassen, einkommentieren

    $$ padd(L) = \{1^l O x | x \in L, l = 2^{|x|} \} $$
\end{definition}


\begin{satz}
    
    Es gilt: 
    $ L \in DTIME(f(s^n)) $, dann ist
    $$ padd(L) \in DTIME(f(n)) $$
    Blase $f$ zeitkonstant und insbesondere $f(n) \geq n$

\end{satz}

\begin{beweis}
    
    Sei $T$ eine deterministische Turingmaschine für $L$ und
    % TODO \leq \in ?
    $DTIME_T(x) \leq \in x f(2^n)$

    Die folgende Maschine $T'$ entscheided $padd(L)$:

    Gegeben Eingabe $y$, schreibe $y + 1^l O x$ und prüfe ob $l = 2^{|x|}$
    geht in Zeit $\bigO(|y|)$

    Aufwand: 
    $c f(2^{|x|}) \leq c * f(|y|)$

    Gesamtaufwand:
    $\leq c * f(|y|) + |y|
    = \bigO(f(|y|))$

    falls $f(n) \geq n$ (zeitkonstruierbar)
\end{beweis}




\begin{satz}
    Umgekehrt gilt auch:

    Wenn $padd(L) \in DTIME(f(n))$
    dann $L \in DTIME(f(s^{n+1}))$
\end{satz}

\begin{beweis}
    Sei $T$ eine Turingmaschine für $padd(L)$
    mit $DTIME_T(y) \leq c f(|y|)$

    Wir bauen eine Maschine für $L$:
    Gegeben Eingabe $x$, bilde $y = 1^{2^{|x|}} O x $.

    Aufwand: $\bigO(2^{|x|})$
    Setze $T$ auf $y$ an.
    Aufwand: $c * f(|y|) = c * f(2^{|x|} + |x| + 1)$

    Gesamtaufwand:
    $\bigO(f(2^{|x|+1}))$
\end{beweis}




\section{Was wenn P = NP}

\begin{satz}
    Folgerung:
    $$\ComplexityClassP = \ComplexityClassNP \Rightarrow \ComplexityClassE = \ComplexityClassNE$$
\end{satz}
\begin{beweis}
    Sei $P = NP$
    und $L \in NE$ und $T$ eine Maschine mit Aufwand $n^{kk}$
    wobei $k$ fest, $n$ Länge der Eingabe.
    
    $L \in NTIME(n^{nk}) = NTIME((2^n)^k)$
    Also $padd(L) \in NTIME(2^k)$
    also $padd(L) \in  NP$ und nach Annahme $padd(L) \in P$

    Also $padd(L) \in DTIME(n^{k'})$
    also $L \in DTIME((2^{n+1})^{k'})
    = DTIME(2^{k'n + k'}) = DTIME(2^{k'n}) \subseteq E$
\end{beweis}



Slogan: Gleichheit von Komplexitätsklassen vererbt sich nach oben.

Mit anderen Paddingfunktion zeigt man ebenso:

$$ \ComplexityClassP = \ComplexityClassNP \Rightarrow \ComplexityClassEXP = \ComplexityClassNEXP $$
$$ \ComplexityClassE = \ComplexityClassNE \Rightarrow \ComplexityClassEXP = \ComplexityClassNEXP $$

Kontrapositiv ausgedrückt:

$$ \ComplexityClassE \neq \ComplexityClassNE \Rightarrow \ComplexityClassP \neq \ComplexityClassNP $$ etc.


Es koennte sein, dass $\ComplexityClassP \neq \ComplexityClassNP$ aber doch $\ComplexityClassE = \ComplexityClassNE$.

Slogan: Trennung von Komplexitätklassen vererbt sich (durch Padding) von oben nach unten.




\section{Sogenannte Effizienz}


$\ComplexityClassP$ wird gemeinhin gleichgesetzt mit ``effizient loesbar''.
    
Wachstumsverhalten:

  $p$ Polynom. $\Rightarrow$ 
  $$ \exists c > 0:   p(2n) \leq c * p(n) $$

  Bei Verdopplung des Inputs wird der Output also ver-$c$-facht.

  Häufig hat eine Bruteforce-Lösung (``stures Durchprobieren'')
  exponentiellen und eine echte algorithmische Lösung hat polynomiellen Aufwand.



% chapter np_und_p (end)
