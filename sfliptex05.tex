% Author: Philipp Moers <soziflip+latex@gmail.com>


\usepackage[utf8]{inputenc}
\usepackage[T1]{fontenc}

\usepackage[ngerman, english]{babel}
% document: \selectlanguage{ngerman}


\usepackage{hyphenat}

\usepackage{xifthen}

% Häkchen mit \checked
\usepackage{wasysym} 

% URLs
% \usepackage[hyphens]{url}
% hyperref loads url internally!
\PassOptionsToPackage{hyphens}{url}
\usepackage{hyperref}

% Uhrzeit mit \currenttime
\usepackage{datetime}

\newcommand{\todo}[1]   {\textcolor{red}{TODO: #1}} 


\usepackage{moreverb}



%%%%%%%%%%%%%%% Font %%%%%%%%%%%%%%%


% \renewcommand*{\familydefault}{\rmdefault}
\renewcommand*{\familydefault}{\sfdefault}
% \renewcommand*{\familydefault}{\ttdefault}


% Achtung, eigener Font im Mathematik Abschnitt!



%%%%%%%%%%%%%%% Unicode characters %%%%%%%%%%%%%%%

\usepackage{newunicodechar}
\usepackage{marvosym}

% \newunicodechar{∈}{\in}
\newunicodechar{∈}{$\in$}
\newunicodechar{€}{$\EURdig$}





%%%%%%%%%%%%%%% Farben %%%%%%%%%%%%%%%

\usepackage[usenames,dvipsnames,svgnames,table]{xcolor}
\definecolor{grau}{gray}{.90}
% \definecolor{orange}{RGB}{255,127,0}
\definecolor{myred}{HTML}{E78356}
\definecolor{myorange}{HTML}{FCD078}
\definecolor{mylightblue}{HTML}{A7CED1}
\definecolor{mylightgreen}{HTML}{9DE66E}
\definecolor{mylightpurple}{HTML}{FF9BF6}
\definecolor{mark}{HTML}{FFDDCC}


%%%%%%%%%%%%%%%	Tabellen %%%%%%%%%%%%%%%

\usepackage{tabularx}
\usepackage{colortbl}
\usepackage{multirow}
\newcolumntype{C}[1]{>{\centering\arraybackslash}m{#1}}


%%%%%%%%%%%%%%%	Layout etc. %%%%%%%%%%%%%%%

% \usepackage{enumerate}
% \usepackage{paralist} % compactenum

%%% IN JEDEM ÜBUNGSBLATT %%%
% \usepackage{sectsty}
% % \allsectionsfont{\large\underline} % überschriftendesign
% \sectionfont{\large\underline} % überschriftendesign
% \subsectionfont{\normalsize}


\usepackage{caption}
\captionsetup{margin=10pt,font=small,labelfont=bf}

\usepackage{courier}

\usepackage{pgf} \pgfmathsetmacro{\dotradius}{0.05}

\usepackage[a4paper]{geometry}

\usepackage{rotating}

\usepackage{setspace} \doublespacing

\renewcommand{\baselinestretch}{1.25}\normalsize
\renewcommand{\arraystretch}{1.0}


%%%%%%%%%%%%%%%	Übungsblatt Header %%%%%%%%%%%%%%%

% \newcommand{\header}[3]{\setlength{\parindent}{0pt}#1\hfill\today\\[12pt]\begin{huge}\textbf{#2}#3\end{huge}\\\rule{1\textwidth}{0.3pt}\\[9pt]}
\newcommand{\header}[3]
{\setlength{\parindent}{0pt}
\hfill\today\\
#1\\[18pt]
\begin{large}\textbf{#2}\end{large}\\[9pt]
\begin{huge}\textbf{#3}\end{huge}\\
\rule{1\textwidth}{0.3pt}\\[6pt]}



%%%%%%%%%%%%%%% Plots %%%%%%%%%%%%%%%

% \usepackage{gnuplottex}




%%%%%%%%%%%%%%%	Zeichnungen %%%%%%%%%%%%%%%

\usepackage{tikz}
\usetikzlibrary{decorations.markings,arrows} 
\usetikzlibrary{shapes,positioning,shadows,arrows,automata}
\tikzstyle{mystate} = [state, minimum size = 1.5cm, node distance = 2.5cm]
\tikzstyle{arrowlink} = [decoration={markings,mark=at position 0.1 with \arrow{triangle 60}}, postaction=decorate, color=black] 
\tikzstyle{nippel} = [circle, fill = myorange, draw, font=\sffamily\bfseries, minimum size=1.5cm]
\tikzstyle{bucket} = [rectangle, draw, font=\sffamily\bfseries, auto, thick, node distance = 1cm, minimum size=1cm]
\tikzstyle{decision} = [diamond, draw, fill=mark, text width=4.5em, text badly centered, node distance=3cm, inner sep=0pt]
\tikzstyle{block} = [rectangle, draw, fill=mark, text width=5em, text centered, rounded corners, minimum height=4em]
\tikzstyle{line} = [draw, -latex'] 
\tikzstyle{bplus}=[rectangle split, rectangle split horizontal,rectangle split ignore empty parts,draw]


% entity relationship diagram
% \usepackage{tikz-er2}
% \tikzstyle{every entity} = [top color=white, bottom color=myorange!30, draw=myorange!50!black!100, drop shadow]
% % \tikzstyle{every weak entity} = [drop shadow={shadow xshift=.7ex, shadow yshift=-.7ex}]
% \tikzstyle{every attribute} = [top color=white, bottom color=mylightblue!20, draw=mylightblue, node distance=1cm, drop shadow]
% \tikzstyle{every relationship} = [top color=white, bottom color=myred!20, draw=myred!50!black!100, drop shadow]
% % \tikzstyle{every isa} = [top color=white, bottom color=green!20, draw=green!50!black!100, drop shadow]


%%%%%%%%%%%%%%% Programmcode %%%%%%%%%%%%%%%

\usepackage{listings}

% \usepackage{listingsutf8} % kannste vergessen, chars müssen in ein byte passen

% replace special characters
\lstset{
  literate={ö}{{\"o}}1
           {Ö}{{\"O}}1
           {ä}{{\"a}}1
           {Ä}{{\"A}}1
           {ü}{{\"u}}1
           {Ü}{{\"U}}1
           {ß}{{\ss}}1
}

\usepackage{framed}
\colorlet{shadecolor}{grau}
\newcommand{\codeline}[1]			{\colorbox{grau}{\texttt{#1}}}
% \newcommand{\codebox}[1]			{\begin{shaded}\texttt{#1}\end{shaded}}
% \newcommand{\codebox}[1]			{\begin{lstlisting}{\texttt{#1}}\end{lstlisting}}
\lstnewenvironment{codebox}[1][] 	{\lstset{		language = #1,
													backgroundcolor = \color{grau},
                                                    % basicstyle = \ttfamily,
                                                    basicstyle = \scriptsize,
                                                    % basicstyle = \tiny,
                                                    numbers = left, % none
                                                    numberstyle = \tiny,
                                                    breaklines = true,
                                                    breakatwhitespace = true,
                                                    escapeinside = {@}{@},
                                                    frame = single,
                                                    tabsize = 2
}}{}
\newcommand{\codefile}[3] 		{\lstinputlisting[	language = #1,
                                                    % extendedchars = true,
                                                    % inputencoding = utf8/utf8,
													backgroundcolor = \color{grau},
                                                    basicstyle = \ttfamily,
                                                    numbers = left, % none
                                                    numberstyle = \tiny,
                                                    breaklines = true,
                                                    breakatwhitespace = true,
                                                    frame = single,
                                                    tabsize = 2,
													% firstline=42
													% lastline=50
													#2]
{#3}}


% JAVASCRIPT SUPPORT
\lstdefinelanguage{JavaScript}{
  keywords={typeof, new, true, false, catch, function, return, null, catch, switch, var, if, in, while, do, else, case, break},
  keywordstyle=\color{blue}\bfseries,
  ndkeywords={class, export, boolean, throw, implements, import, this},
  ndkeywordstyle=\color{darkgray}\bfseries,
  identifierstyle=\color{black},
  sensitive=false,
  comment=[l]{//},
  morecomment=[s]{/*}{*/},
  commentstyle=\color{purple}\ttfamily,
  stringstyle=\color{red}\ttfamily,
  morestring=[b]',
  morestring=[b]"
}
\lstset{
   language=JavaScript,
   backgroundcolor=\color{lightgray},
   extendedchars=true,
   basicstyle=\footnotesize\ttfamily,
   showstringspaces=false,
   showspaces=false,
   numbers=left,
   numberstyle=\footnotesize,
   numbersep=9pt,
   tabsize=2,
   breaklines=true,
   showtabs=false,
   captionpos=b
}






%%%%%%%%%%%%%%%	Mathematik etc. %%%%%%%%%%%%%%%

% \usepackage{MnSymbol}

% conflict with wasysym package
\let\iint\relax
\let\iiint\relax

% \usepackage{amsmath} % not needed with mathtools
\usepackage{amssymb}

\usepackage{mathtools}


\usepackage{mathcomp}


% cool mathmode font
\usepackage{eulervm}
% ... with global font
\usepackage{mathpazo}


% --- Einheiten: ---
% (Syntax: \SI{wert}{einheit})
% \usepackage{siunitx}
% \sisetup
% {
%  	alsoload=binary,    % Binäre Einheiten (\bit, \byte)
%   % alsoload=synchem,   % Chemische Einheiten (\mmHg, \molar, \torr, ...)
%   % alsoload=astro,     % Astronomische Einheiten (\parsec, \lightyear)
%   % alsoload=hep,       % Einheiten der Hochenergie-Physik (\clight, \eVperc)
%   % alsoload=geophys,   % Einheiten der Geophysik (\gon)
%   % alsoload=chemeng    % Einheiten der chem. Verfahrenstechnik (\gmol, \kgmol, ...)
% 	unitsep=thin, valuesep=space
% }

% \newcommand{\lor}				{\ensuremath{\vee}}
% \newcommand{\land}			{\ensuremath{\wedge}}
\newcommand{\lxor}				{\ensuremath{\oplus}}

\newcommand{\setunion} 			{\ensuremath{\cup}}
\newcommand{\setintersection} 	{\ensuremath{\cap}}
 
\newcommand{\R}					{\ensuremath{\mathbb{R}}}
\newcommand{\N}					{\ensuremath{\mathbb{N}}}
\newcommand{\Z}					{\ensuremath{\mathbb{Z}}}

\newcommand{\bigO}				{\ensuremath{\mathcal{O}}} 			% big-O notation/symbol
