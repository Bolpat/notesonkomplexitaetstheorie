%!TEX root = 0-main.tex

% Author: Philipp Moers <soziflip@gmail.com> 



\datum{26.11.15}

\chapter{Platzkomplexität} % (fold)
\label{cha:platzkomplexitaet}
    



\section{Platzkonstruierbare Funktionen}


\begin{definition}
    
    Eine Funktion $s(n)$ heißt \definiere{platzkonstruierbar} genau dann, wenn eine deterministische Turingmaschine existiert, die bei Eingabe $0^n$ genau $s(n)$ Bandfelder beschreibt und dann hält.

\end{definition}

\begin{beispiel}

    Alle Polynome mit Koeffizienten $\in \Q^+$, die Wurzelfunktion, die Logarithmus-Funktion, die Potzenfunktion usw. sind platzkonstruierbar.
    
\end{beispiel}




\section{Platzverbrauch einer Turingmaschine}

\begin{definition}

    Der \definiere{Platzverbrauch einer Turingmaschine} (deterministisch oder nichtdeterministisch) bei Eingabe $x$ ist 

    \begin{itemize}

        \item \textbf{erste Definition}\\ 
        die Größe des beschriebenen Teils aller Bänder am Ende der Berechnung. (Mit dieser Definition ist der Platzverbrauch stets $\geq |x|$).

        \item \textbf{zweite Definition}\\ 
        die Endgröße aller anderen Bänder, wobei das Eingabeband nicht überschrieben werden darf.

    \end{itemize}

    Die zweite Definition ist Standard,w enn sublineare Platzschranken betrachtet werden, zum Beispiel $\log(n)$. Oberhalb von $\bigO(n)$ sind die beiden Definitionen äquivalent.
    


    Notation: $DSPACE_M (x)$ und $NSPACE_M (x)$

    $DSPACE_M (s(n)) = \{   L  \ |\   \exists DTM M : L = L(M) \land DSPACE_M(x) = \bigO(s(|x|)) \}$

    $NSPACE_M (s(n)) = \{   L  \ |\   \exists DTM M : L = L(M) \land NSPACE_M(x) = \bigO(s(|x|)) \}$

    $PSPACE = \bigcup_{k \geq 0} DSPACE(n^k) $ (polynomieller Platz)

    $LINSPACE = DSPACE(n)$

    $LOGSPACE = DSPACE(\log n)$ (auch als $L$ bezeichnet)

    $NLOGSPACE = NSPACE(\log n)$ (auch als $NL$ bezeichnet)


\end{definition}


\begin{beispiel}
    
    STCONN (Erreichbarkeit in gerichteten Graphen) ist $\in NLOGSPACE$ (rate Pfad) und $\in LINSPACE$ (Tiefensuche/Breitensuche)

\end{beispiel}


Es gibt eine triviale, aber wissenswerte Beziehung zwischen Zeit- und Platzkomplexität:

% $$ NSPACE(s(n)) \subseteq DTIME (2^{\bigO(s(n))}) $$      % not sure
$$ DSPACE(s(n)) \subseteq DTIME (2^{\bigO(s(n))}) $$

% Alle Berechnungsfolgen können aufgezählt werden.
Hat die Berechnung nach $2^{c*s(n)}$ Schritten nicht geendet, so kann abgebrochen werden wegen Wiederholung einer globalen Konfiguration. (Tatsächlicher Platzverbrauch $\leq c * s(n)$)



$$ DTIME(t(n)) \subseteq DSPACE (t(n)) $$

Mehr Platz als Laufzeit kann nicht angefordert werden.





\begin{satz}

    Für deterministische Einband-Turingmaschinen $T$ gilt:

    $DTIME_T(x) = \bigO(t(|x|) \Longrightarrow  L(T) \in DSPACE(\sqrt{t(n)})$

    
\end{satz}

Für Mehrband-Turingmaschinen gibt es einen ähnliches Satz, bei dem der Platz allerdings etwas größer ist. Dass er für Einband-Turingmaschinen so gut ist, ist gewissermaßen kurios.





\section{Platzhierarchiesatz}

\begin{satz}
    
    ``Echt mehr Platz hilft auch mehr.''

    \textit{Für genaue Aussage und Beweis siehe z.B. Papadimitrion}
\end{satz}

Wichtige Konsequenz:

$$ LOGSPACE \subset PSPACE $$

$$ LOGSPACE \subseteq NLOGSPACE \subseteq \ComplexityClassP \subseteq \ComplexityClassNP \subseteq PH \subseteq PSPACE $$

Von jeder dieser Inklusionen ist unbekannt, ob sie echt sind. Mindestens eine muss aber echt sein.






\section{Zusammenhänge von Platzkomplexitätsklassen}


\begin{satz}
    
    \definiere{Satz von Savitch}

    Für eine platzkonstruierbare Funktion $s(n) \geq \log(n)$ ist \\
    $ NSPACE(s(n)) \subseteq DSPACE(s(n)^2)$

    (Vergleiche
    $ NTIME(t(n)) \subseteq DTIME(2^{\bigO(t(n))})$
    )

\end{satz}

\begin{beweis}
    
    Sei eine nichtdeterministische Turingmaschine $T$ gegeben mit Platzbedarf $S = c * s(|x|)$ bei Eingabe $x$. Wir betrachten eine Kodierung der globalen Konfigurationen von $T(x)$ durch Wörter der Länge $S$ und o.B.d.A gebe es exakt eine akzeptierende Endkonfiguration $s_{ACC}$. (Alle Bänder am Ende löschen, d.h. mit $0$ überschreiben.)

    $
    x \in L(T) 
        \Longleftrightarrow 
    s_{ACC} \text{ von } s_{INI} \text{ aus in } \leq 2^S \text{ Schritten erreichbar}
    $

    Hier steht $s_{INI}$ für die Startkonfiguration bei Eingabe $x$. $2^S$ ist die Gesamtzahl der Konfigurationen.

    Das heißt $s_{ACC}$ ist von $s_{INI}$ aus im Graphen der Konfigurationen erreichbar (Spezialfall von STCONN).

\end{beweis}


\textbf{Notation:}

$s \rightarrow_T s'$: $s'$ ist 1-Schritt-Folgekonfiguration von $s$ in $T$ und kann in $LOGSPACE$ entschieden werden.

$REACH(s, s')$: $s \rightarrow^\ast s'$ ($s'$ ist von $s$ erreichbar)

$REACH(s, s', i)$: $s \rightarrow^{\leq 2^i} s'$ ($s'$ ist von $s$ in weniger als $2^i$ Schritten erreichbar)


$ x \in L(T) \Longleftrightarrow REACH(s_{INI}, s_{ACC}, S)$

Es gilt 
$$REACH(s, s', 0) \Longleftrightarrow s = s' \lor s \rightarrow_T s'$$
\hspace{4cm}($2^0 = 1$)
$$REACH(s, s', i+1) \Longleftrightarrow \exists \check{s} : REACH(s, \check{s}, i) \land REACH(\check{s}, s', i)$$ 
\hspace{4cm}($2^{i+1} = 2 * 2^i$)\\
Dies liefert eine rekursive Implementierung von $REACH(s,s',i)$ \\
\hspace{4cm}($\exists \check{s} \leadsto \text { for } \check{s} \in \text{ globale Konfigurationen}$)


Der Rekursionsstack hat Tiefe $S$ (Toplevel-Aufruf $REACH(s_{INI}, s_{ACC}, S)$).
\\
Jeder Activationrecord hat Größe $\bigO(S)$ genauer gesagt $2 S$ für die beiden Parameter $s, s', \log(S)$ für die Parameter $i$. Wenn gewünscht noch ein weiteres $S$ für die for-Schleife.


Die Gesamtgröße des Stacks ist beschränkt durch $S * \bigO(S) = \bigO(S^2)$.


Historisch wurde zunächst gezeigt, dass STCONN in $DSPACE(\log(n)^2)$ liegt. Der Satz von Savitch kann auch hieraus abgeleitet werden.









\datum{30.11.15}




\begin{satz}

    \definiere{Satz von Immerman-Szelepcsényi}

    Sei $s(n) \geq \log(n)$.

    Dann ist $NSPACE(s(n)) = co\text{--}NSPACE(s(n))$

    Wichtiger Spezialfall: $co\text{--}STCONN \in NSPACE(\log(n)) = NL = NLOGSPACE$
    \\
    Die allgemeine Behauptung kann aus diesem Spezialfall leicht gefolgert werden (durch den Graph der globalen Konfiguration).


\end{satz}

\begin{beweis}

    Es sei eine nichtdeterministische Turingmaschine $T$ vorgelegt und $NSPACE_T(x) \leq c * s(|x|)$. Wir müssen eine nichtdeterministische Turingmaschine $T'$ konstruieren, sodass $NSPACE_{T'}(x) = \bigO(s(|x|))$ und $x \in L(T') \Leftrightarrow x \notin L(T)$.
    \\
    Es existiert akzeptierende Berechnung von $T'$ auf x genau dann, wenn alle Berechnungen von $T$ auf $x$ verwerfen.

    Sei $x$ fixiert und o.B.d.A $\Sigma = \{0,1\}$.
    \\
    Schreibe $S_{INI}$ für die globale Startkonfiguration von $T$ auf $x$
    und $S_{ACC}$ für die (o.B.d.A. einzige) akzeptierende globale Konfiguration.
    Weiter sei $S = c' *  s(|x|)$ so gewählt, dass alle globalen Konfigurationen durch $0/1$-Strings der Länge $S$ kodiert werden.
    \\
    $s \rightarrow_T s'$ bedeute, dass $s'$ in einem Schritt aus $s$ hervorgehen kann (Das kann in Platz $\log(S)$ entschieden werden).
    \\
    $T'$ soll nun $x$ akzeptieren genau dann, wenn kein Pfad (der Länge $2^S$) von $s_{INI}$ zu $s_{ACC}$ existiert.
    (Anzahl der globalen Konfigurationen ist kleiner als $2^S$. Ein einziger Pfad braucht, wenn er voll ausgeschrieben wird, schon Platz $S * 2^S \notin \bigO(s(|x|))$.)

    \textbf{Vorbemerkung:}\\
    Nehmen wir an, dass die Anzahl $N$ der von $s_{INI}$ aus erreichbaren globalen Konfigurationen bekannt ist bzw. berechnet werden kann.

    Wir zählen der Reihe nach alle globalen Konfigurationen auf (geht mit Platz $\bigO(S)$) und raten für jede von denen einen Pfad von $S_{INI}$ dorthin. Durch Mitführen eines Zählers haben wir am Ende der Aufzählung die Anzahl derjenigen Knoten, für die das gelungen ist.

    \begin{codebox}[javascript]
function A(...) {
    cnt = 0;
    for (s in globalConfigs) {
        pfad = guessPath();
        if (pfad.endsAt(s))
            cnt++;
        if (s = s_acc)
            return "reject";
    }
    if (cnt == N)
        return "accept";
    else
        return "reject";
}

function guessPath() {
    s = s_ini;
    for (i = 1; i <= 2^S; i++) {
        sX = guessNonDet({0,1}^S);
        if (s2 -> sX) {
            s2 = sX;
        } else {
            return "reject";
        }
        b = guessNonDet({0,1});
        if (b) {
            break;
        }
    }
}
    \end{codebox}

    Falls $N$ die Anzahl der von $s_{INI}$ aus erreichbaren globalen Konfigurationen ist, so \underline{kann} A akzeptieren genau dann, wenn $s_{ACC}$ von $s_{INI}$ unerreichbar ist. 
    \\
    Begründung ``$\Rightarrow$'': Falls A akzeptiert, dann ist $s_{ACC}$ tatsächlich unerreichbar, weil alle $N$ erreichbaren Konfigurationen in der for-Schleife als solche erkannt wurden und $s_{ACC}$ nicht unter ihnen war.
    \\
    Begründung ``$\Leftarrow$'': Falls $s_{ACC}$ unerreichbar ist, so kann A akzeptieren, indem bei jedem der von $s_{INI}$ aus erreichbaren s tatsächlich ein entsprechender Pfad geraten wird.


    Grobe Struktur des Algorithmus für $T'$:
    \begin{codebox}[javascript]
N = 1;
for (i = 1 ... 2^S) {
    // Invariante: Anzahl der von s_ini aus in weniger als i-1 Schritten erreichbaren Konfigurationen ist gleich N
    updateN();
}
A();
    \end{codebox}
    Der Block \codeline{updateN()} wird selbst Nichtdeterminismus enthalten, in dem Sinne, dass die gesamte Berechnung verwerfend abgebrochen werden kann. Passiert das nicht, dann ist $N$ korrekt aktualisiert und bei passender Wahl der nichtdeterministischen Entscheidungen, passiert das auch.

    \begin{codebox}[javascript]
function updateN() {
    cnt = 0;
    for (s in globalConfigs) {
        reachable = false;
        cnt2 = 0;
        // alle N Stück die von s_ini aus in weniger als i-1 Schritten erreichbar sind, aufzählen
        for (sCheck in globalConfigs) {
            pfad = guessPath(); 
            if (pfad.endsAt(sCheck)) {
                cnt2++;
                if (sCheck == s || sCheck -> s)
                    reachable = true;
            }
        }
        if (cnt2 != N)
            return "reject";
        else if (reachable)
            cnt++;
    }
    N = cnt;
}
    \end{codebox}

    Am Ende von \codeline{updateN} hat entweder N den korrekten Wert oder es wurde verworfen.
    Es ist möglich, die nichtdeterministischen Entscheidungen so zu treffen, dass der korrekte Wert geliefert wird, sodass nicht verworfen wird.


\end{beweis}










\datum{03.12.15}



\section{Noch ein Satz von Cook}


\begin{definition}

    $NSPACE(s(n)) + STACK$

    Intuitiv ist das alles, was mit $\bigO(s(n))$ platzbeschränkten Arbeitsbändern und einem unbeschränkten Stack berechnet werden kann.

    Eine mögliche Formalisierung: Turingmaschine mit Stack hat zusätzlich ein Stackalphabet $\Gamma$ und ein besonderes Symbol $A \in \Gamma$. Jedes ``Quintupel'' enthält eine zusätzliche Komponente, die Stack-Aktion, eine der folgenden drei:
    \begin{itemize}
        \item IDLE (Keller bleibt unverändert)
        \item POP (oberstes Kellersymbol wird entfernt)
        \item PUSH ($x \in \Gamma$ auf den Keller legen)
    \end{itemize}
    Außerdem ist das jeweils oberste Kellersymbol lesbar.
    \\
    Die Maschine wird mit Kellerinhalt $\bigAoverlinesqcup$ gestartet und hält per Definition, wenn der Keller leer wird, d.h. wenn $A$ gePOPt wird.

    Formal:
    \\
    $\delta = Q \times (\Sigma^k \times \Gamma) \times Q \times \Sigma^k \times S$
    \\
    wobei $S$ die Menge der Stack-Aktionen ist.


\end{definition}


\begin{beispiel}
    
    Der Beweis des Satzes von Savitch zeigt, dass STCONN $\in DSPACE(\log(n) + STACK)$ und $NSPACE(s(n)) \subseteq DSPACE(s(n) + STACK)$ ist.



\end{beispiel}



\begin{satz}
    
    $s(n) \geq \log(n) \Longrightarrow NSPACE(s(n)) + STACK = DTIME(2^{\bigO(s(n))})$

\end{satz}

\begin{beweis}

    \begin{itemize}

        \item ``$\subseteq$'':

        Sei $T$ eine zunächst deterministische durch $s(n)$ platzbeschränkte Turingmaschine mit Stack.
        \\
        Wir fixieren eine Kodierung der ``Bandkonfigurationen'' von $T$. Diese Bandkonfigurationen beinhalten den Zustand und gesamten Inhalt aller Arbeitsbändern und Kopfpositionen. Bei Eingabe der Größe $n$ haben die so kodierten Bandkonfigurationen die Größe $\bigO(s(n))$, also konkret $c * s(n)$ für ein festes $c$.
        \\
        Sei nun eine Eingabe fixiert. $B$ sei die Menge der Bandkonfigurationen der Größe $c * s(|x|)$ und $\Gamma$ das Stackalphabet.
        \\
        Wir definieren die partielle Funktion $f: B \times \Gamma \longrightarrow B: f(b, x) = b'$
        \\
        $b'$ ist die Bandkonfiguration, die erreicht wird, wenn man $T$ in $b$ und mit Kellerinhalt $\bigxsqcup$ startet und so lange rechnet, bis $X$ zum ersten Mal gePOPt wird. 


        \textbf{Bemerkung:}\\
        $x \in L(T) \Longleftrightarrow f(b_{INI}, A) = b'$ und $b'$ akzeptierende Endkonfiguration ist. O.B.d.A ist $b'$ eindeutig.

        \textbf{Bemerkung:}\\
        Streng genommen sollte man schreiben $f_x(b, X)$ oder $f(x, b, X)$, da die Eingabe $x$ in die Definition von $f$ einfließt.

        $f$ gestattet folgende rekursive Definition: 

        $f(b, X) = 
        \begin{cases}
        b'                  & , \text{POP}     \\
        f(b', X)            & , \text{IDLE}   \\
        f(f(b', Y), X)      & , \text{PUSH(Y)}
        \end{cases}
        $

        wobei $b'$ die auf $(b,X)$ unmittelbar folgende Bandkonfiguration ist.

        Statt nun $f(b_{INI}, A)$ durch Rewriting auszuwerten, was letztendlich der Berechnung von $T$ entspräche, tabulieren wir die gesamten $f$-Werte systematisch mit dynamischer Programmierung. Die Anzahl der Bandkonfigurationen ist $2^{c*s(|x|)}$. Damit gibt es ingesamt $\bigO(2^{c*s(|x|)})$, o.B.d.A $2^{c*s(|x|)})$ viele $f$-Aufrufe. Eine Wertetabelle für $f$ kann also in Zeit $2^{\bigO(s(|x|))}$ komplett ausgefüllt werden.
        \\
        Gesamtrechenzeit: $\bigO((2^{c*s(|x|)})^2) = 2^{\bigO(s(|x|))}$ 

        
        Wenn die Maschine nichtdeterministisch ist, dann nehmen wir statt $f$ die Funktion $F: B \times \Gamma \times B \longrightarrow bool:\\
        F = 
        \begin{cases}
         true        & \text{falls es Berechnung von } b \text{ nach } b' \text{ gibt, an deren Ende } X \text{ erstmals gePOPt wird}       \\
         false       & \text{sonst}     \\
        \end{cases}
        $

        $F$ kann ganz genauso mit dynamischer Programmierung ausgewertet werden.



        Sei umgekehrt $T$ eine deterministische Turingmaschine mit Zeitschranke $2^{c*s(n)}$.
        \\
        Die Kopfpositionen aller Köpfe könenn durch String der Länge $\bigO(s(n))$ kodiert werden (Binärkodierung der numerischen Positionen).
        $a$ mit $|a| = c' * s(n)$ kodiere diese Positionen. 
        \\
        Ebenso können Zeitpunkte in der Berechnung durch String dieser Länge kodiert werden.

        Wir definieren bei fester Eingabe $x$ folgende rekursive Funktionen: 
        \begin{itemize}
            \item 
            $band(t, a) \in \Sigma^k$ als Bandinhalt zur Zeit $t$ an den Positionen $a$.
            \\
            $band(0,   a) = \text{ Inhalt der initialisierten Bänder an den geforderten Positionen }$
            $band(t+1, a) = \dots zustand(t) \dots band(t,a) \dots kopf(t)$

            \item 
            $zustand(t) \in Q$
            \\
            $zustand(0) = q_0$

            \item 
            $kopf(t)    = \text{ Positionen aller Köpfe zur Zeit } t$
            \\
            $kopf(t+1)  = \dots kopf(t) \dots band(t,a) \dots zustand(t) $

        \end{itemize}

        Die klassische rekursive Auswertung dieser Funktionen beziehungsweise des Toplevel-Aufrufs $zustand(2^{c*s(|x|)}) \in \{ q_{ACC}, q_{REG} \} $ kann auf einer $DSPACE(s(n)) + STACK$ Maschine simuliert werden.

        % TODO der beweis ist noch nicht fertig oder

\end{itemize}    

    
\end{beweis}

\datum{7.12.2015}
\definiere{In logarithmischem Platz berechenbare Funktion}
Eine Funktion $f: \Sigma* \Rightarrow \Sigma*$ ist in logarithmischem
Platz berechebar, wenn es eine DTM gibt, die bei Eingabe x den
Funktionswert $f(x)$ auf ein gesondertes Ausgabeband schreibt.

Das Ausgabeband darf nicht gelesen werden, das Eingabebnad (wie immer
bei Space) nicht beschrieben werden.

Die Arbeitsbänder sind $\bigO(log|x|)$ platzbeschränkt.

Äquichvalente Chrakterisierung: Maschine schreibt das n-te Symbol auf
ein besonderes Band (oder signalisiert es durch Einnehmen eines
bestimmtne Zustandes), wenn n in Binärkodierung auf ein besonderes
Band geschreiben wird.

\textbf{Bemerkung}
$2^{c*log(n)}=n^c$
$ f \in FP \Rightarrow |f(x)| = \bigO(n^c)$ für ein c,
d.h. polynomiell beschränkt und auch $DTIME_T(x) \in \bigO(n^c)$
bei passenden T, welches f berechnet.

\textbf{Bemerkung $FL \subseteq FP$}
\begin{satz}
FC ist unter KOmposition abgeschlossen.
$f,g \in FL \Rightarrow g \times f \in FL$
\end{satz}

Gegeben Maschienen $T_f$ und $T_g$ im zweiten Format (n-tes Symbold on
$f(x)$ wird bei Eingabe von x und n ``on demand'' berechnet.

Maschine $T_{g \times f}$ für $g \times f $ wird ebenfalls im
2. Format konstruiert. Sie simuliert auf der äußeren Ebene die
Maschine $T_g$.  Jede Leseanfrage der Einfabe wird in eine Berechnung
von $T_f$ auf x an der entsprechnen Ausgabeposition übersetzt.

\definiere{NL-hart, P-hart, NL-vollständig, P-vollständig}
\begin{itemize}
\item X ist NL-hart, wenn für jedes $X'\in NL$ gilt $X' \leq_L X$
\item X ist P-hart, wenn für jedes $X'\in P$ gilt $X' \leq_L X$
\item X ist NL- bzw P-vollständig, wenn X ist NL-/P-hart und $X\in NL/P$
\end{itemize}
Hierbei bedeuted $X' \leq_L X$, dass $f\in FL$ existiert, mit $x \in X' \Leftrightarrow f(x) \in X$
NB: $X' \leq_L X \Rightarrow x\leq_P X, bzw. X\leq X'$

\begin{satz}Das Problem STCONN (Erreichbarkeit im gerichteten Graphen) ist NL-vollständig.
\end{satz}

\begin{beweis}
$STCON \in NL$ \checkmark

\paragraph{NL-hart} Sei T irgendeine NL-Maschiene. $x \in ((T) E)$
es exestirt ein PFad von $a_{INI}$ nach $a_{ACC}$ im Graphen, dessen
Knoten die globale Konfig. von T bei Eingabe x sind $a_{INI}$ die
Startkonfiguration und $a_{ACC}$ die (oBdA eindeutige)akzeptierende Eindkonfiguration ist. Die Übersetzung von x in dem Graphen kann offensichtlich durch eine ``Fl-Maschine'' im 1. Format geschehen.
\end{beweis}

\definiere{Hornklausel}
Eine Klausel (Disjunktion von Literalen) mit höchstens einem positiven
Literal heißt Hornklausel. Hornklauseln können logische äquivalent in
den Formaten:


%das v ist hier gemeint
\begin{itemize}
\item $P_1, P_2, \dots P_k \rightarrow q$ 
\item d. h. $\neg p_1 \lor \neg p_2 \lor \dots \neg p_k \lor q$

\item oder $P_1, P_2, \dots P_k \rightarrow \bot$
\item d.h. $\neg p_1 \lor \neg p_2 \lor \dots \neg p_k$
\end{itemize}
geschrieben werden.

Das Problem Horn.
\paragraph{Gegeben}
Eine Menge von Hornklauseln H.
\paragraph{Gefragt}
Ist H unerfüllbar?

%-----

Eine Hornklausel der Form $\rightarrow q$ k = 0 heißt Faktum.
Eine Hornklausel der Form $\rightarrow \bot$ k = 0 heißt Goal.

In der Rgel entfällt H genau ein ``Goal'', $p\rightarrow \bot$
H ist dann unerfüllbar, $Leftrightarrow$ p aus den übrigen Klausen hergeleitet werden kann. $H \vdash p$ hierbei ist die Herleitbarkeit wie folgt definiert.
\begin{itemize}
\item $\rightarrow q \in H$ dann $H+g$ alle Fankten sind herleitbar
\item $P_1, P_2, \dots P_k \rightarrow q \in H$ und
$H+P_1, H+P_2, \dots H+P_k$ dann auch $H+g$
\end{itemize}

NB: eine Horn-Klausel Menge ohne Goals ist immer erfüllbar.

NB: Eine Hornklauselmenge mit mehreren Goals ist unerfüllbar
$\Leftrightarrow$ mindestens ein Goal herleibar ist.

\begin{satz}
$Horn \in P$ Wir benutzen eine dynamsiche Menge S, die am Ende alle
herleitbaren Variablen enthalten soll. $S:=\emptyset$
\end{satz}

%Pseudocodes
\begin{codebox}[javascript]
while(!done)
  s' =  s
  for p_1, \dots, p_k  \rightarrow q \in H
    if{p1, \dots, p_k} \seteq S' then
      S' := S' \u {q}
    done := S' = S''; S := S' 
return ``es existiert P \rightarrow \bot \in H mit q \subseteq S''
\end{codebox}

Dieses Vorgehen läuft in polynomieller Zeit, da die Zahl der
Durchläufe durch die Zahl der Variablen beschränkt ist, aber nicht
offensichtlich in log. Platz, da die dynamische Menge lineren Platz
benötigt.

\begin{beispiel}
\begin{itemize}
\item $ \rightarrow A $
\item $ A \rightarrow B $
\item $ F\rightarrow F $
\item $ B,C \rightarrow D $
\item $ A,B \rightarrow C $
\item $ A,D \rightarrow E $
\item $ E\rightarrow \bot $
\end{itemize}
$S:\emptyset ,{A},{A,B},{A,B,C},{A,B,C,D},{A,B,C,D,E}$
\end{beispiel}

\begin{satz}
Horn ist P-Vollständig.
\end{satz}

\begin{beweis}
Genauo wie Satz Cock(Sat ist NP-vollständig), im Fall einer
deterministschen Maschine entstheen bei der Übersetzung nur
Horn-Klauseln.
\paragraph{Details}
Sei DTM T gegeben, Eingabe x $DTIME_T(x) \subseteq p(|x|)$
\paragraph{Variablen}
\begin{description}
\item[zust(t,g)] Zustand zu Zeit t ist g
\item[band(t,i,x)] Bandinhalt zur Zeit t an position i ist
\item[kopf(t,i)] Kopf an Pos i zur Zeit t
\end{description}

Klauseln z.B.
\begin{itemize}
\item $band(t,i,x), Kopf(t,i') \rightarrow band(t+1, i,x)$
\item $band(t,i,x), Kopf(t,i), Zust(t,q) \rightarrow band(t+1, i, x')$
\item $band(t,i,x), kopf(t,i)$
\item $Zustand(t,q) \rightarrow Zustand(t+1, g')\dots$
\end{itemize}
falls jeweils S(g,x) = (g', x', m)

Eingabe x wird akzeptiert $\Leftrightarrow$ Zustand(P|x|), $q_{ACC}$)
herleibar ist um Goal $Zustand(P(|x|), q_{ACC}) \rightarrow \bot$
hinzunehmen.  Die Übersetzung von T, x in diese Klauselmenge kann mit
FL-Maschine durchgeführt werden. Also $L(T) \leq_{L}$ HORN.
\end{beweis}

Das bereits genannte offne Problem $NL \subsetneq P$ NL echt in P enthalten?
kann also umformuliert werden, innn die Frage, ob $HORN \in NL$

\paragraph{Markieren von Klauseln}
Auf ein Fakt darf jederzeit eine Marke gelegt werden. Liegen auf allen
Prämissen $p_1, p_2, \dots, p_k$ einer Klausel
$p_1, p_2, \dots, p_k \rightarrow$ schon Marken,
dann darf q mit einer Marke beheftet werden.
chapter platzkomplexitaet (end)
