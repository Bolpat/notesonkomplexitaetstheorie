%!TEX root = 0-main.tex

% Author: Philipp Moers <soziflip@gmail.com> 



\chapter{Platzkomplexität} % (fold)
\label{cha:platzkomplexitaet}
    



\section{Platzkonstruierbare Funktionen}


\begin{definition}
    
    Eine Funktion $s(n)$ heißt \definiere{platzkonstruierbar} genau dann, wenn eine deterministische Turingmaschine existiert, die bei Eingabe $0^n$ genau $s(n)$ Bandfelder beschreibt und dann hält.

\end{definition}

\begin{beispiel}

    Alle Polynome mit Koeffizienten $\in \Q^+$, die Wurzelfunktion, die Logarithmus-Funktion, die Potzenfunktion usw. sind platzkonstruierbar.
    
\end{beispiel}




\section{Platzverbrauch einer Turingmaschine}

\begin{definition}

    Der \definiere{Platzverbrauch einer Turingmaschine} (deterministisch oder nichtdeterministisch) bei Eingabe $x$ ist 

    \begin{itemize}

        \item \textbf{erste Definition}\\ 
        die Größe des beschriebenen Teils aller Bänder am Ende der Berechnung. (Mit dieser Definition ist der Platzverbrauch stets $\geq |x|$).

        \item \textbf{zweite Definition}\\ 
        die Endgröße aller anderen Bänder, wobei das Eingabeband nicht überschrieben werden darf.

    \end{itemize}

    Die zweite Definition ist Standard,w enn sublineare Platzschranken betrachtet werden, zum Beispiel $\log(n)$. Oberhalb von $\bigO(n)$ sind die beiden Definitionen äquivalent.
    


    Notation: $DSPACE_M (x)$ und $NSPACE_M (x)$

    $DSPACE_M (s(n)) = \{   L  \ |\   \exists DTM M : L = L(M) \land DSPACE_M(x) = \bigO(s(|x|)) \}$

    $NSPACE_M (s(n)) = \{   L  \ |\   \exists DTM M : L = L(M) \land NSPACE_M(x) = \bigO(s(|x|)) \}$

    $PSPACE = \bigcup_{k \geq 0} DSPACE(n^k) $ (polynomieller Platz)

    $LINSPACE = DSPACE(n)$

    $LOGSPACE = DSPACE(\log n)$ (auch als $L$ bezeichnet)

    $NLOGSPACE = NSPACE(\log n)$ (auch als $NL$ bezeichnet)


\end{definition}


\begin{beispiel}
    
    STCONN (Erreichbarkeit in gerichteten Graphen) ist $\in NLOGSPACE$ (rate Pfad) und $\in LINSPACE$ (Tiefensuche/Breitensuche)

\end{beispiel}


Es gibt eine triviale, aber wissenswerte Beziehung zwischen Zeit- und Platzkomplexität:

% $$ NSPACE(s(n)) \subseteq DTIME (2^{\bigO(s(n))}) $$      % not sure
$$ DSPACE(s(n)) \subseteq DTIME (2^{\bigO(s(n))}) $$

% Alle Berechnungsfolgen können aufgezählt werden.
Hat die Berechnung nach $2^{c*s(n)}$ Schritten nicht geendet, so kann abgebrochen werden wegen Wiederholung einer globalen Konfiguration. (Tatsächlicher Platzverbrauch $\leq c * s(n)$)



$$ DTIME(t(n)) \subseteq DSPACE (t(n)) $$

Mehr Platz als Laufzeit kann nicht angefordert werden.





\begin{satz}

    Für deterministische Einband-Turingmaschinen $T$ gilt:

    $DTIME_T(x) = \bigO(t(|x|) \Longrightarrow  L(T) \in DSPACE(\sqrt{t(n)})$

    
\end{satz}

Für Mehrband-Turingmaschinen gibt es einen ähnliches Satz, bei dem der Platz allerdings etwas größer ist. Dass er für Einband-Turingmaschinen so gut ist, ist gewissermaßen kurios.





\section{Platzhierarchiesatz}

\begin{satz}
    
    ``Echt mehr Platz hilft auch mehr.''

    \textit{Für genaue Aussage und Beweis siehe z.B. Papadimitrion}
\end{satz}

Wichtige Konsequenz:

$$ LOGSPACE \subset PSPACE $$

$$ LOGSPACE \subseteq NLOGSPACE \subseteq \ComplexityClassP \subseteq \ComplexityClassNP \subseteq PH \subseteq PSPACE $$

Von jeder dieser Inklusionen ist unbekannt, ob sie echt sind. Mindestens eine muss aber echt sein.



\begin{satz}
    
    \definiere{Satz von Savitch}

    Für eine platzkonstruierbare Funktion $s(n) \geq \log(n)$ ist \\
    $ NSPACE(s(n)) \subseteq DSPACE(s(n)^2)$

    (Vergleiche
    $ NTIME(t(n)) \subseteq DTIME(2^{\bigO(t(n))})$
    )

\end{satz}

\begin{beweis}
    
    Sei eine nichtdeterministische Turingmaschine $T$ gegeben mit Platzbedarf $S = c * s(|x|)$ bei Eingabe $x$. Wir betrachten eine Kodierung der globalen Konfigurationen von $T(x)$ durch Wörter der Länge $S$ und o.B.d.A gebe es exakt eine akzeptierende Endkonfiguration $s_{ACC}$. (Alle Bänder am Ende löschen, d.h. mit $0$ überschreiben.)

    $
    x \in L(T) 
        \Longleftrightarrow 
    s_{ACC} \text{ von } s_{INI} \text{ aus in } \leq 2^S \text{ Schritten erreichbar}
    $

    Hier steht $s_{INI}$ für die Startkonfiguration bei Eingabe $x$. $2^S$ ist die Gesamtzahl der Konfigurationen.

    Das heißt $s_{ACC}$ ist von $s_{INI}$ aus im Graphen der Konfigurationen erreichbar (Spezialfall von STCONN).

\end{beweis}


\textbf{Notation:}

$s \rightarrow_T s'$: $s'$ ist 1-Schritt-Folgekonfiguration von $s$ in $T$ und kann in $LOGSPACE$ entschieden werden.

$REACH(s, s')$: $s \rightarrow^\ast s'$ ($s'$ ist von $s$ erreichbar)

$REACH(s, s', i)$: $s \rightarrow^{\leq 2^i} s'$ ($s'$ ist von $s$ in weniger als $2^i$ Schritten erreichbar)


$ x \in L(T) \Longleftrightarrow REACH(s_{INI}, s_{ACC}, S)$

Es gilt 
$$REACH(s, s', 0) \Longleftrightarrow s = s' \lor s \rightarrow_T s'$$
\hspace{4cm}($2^0 = 1$)
$$REACH(s, s', i+1) \Longleftrightarrow \exists \check{s} : REACH(s, \check{s}, i) \land REACH(\check{s}, s', i)$$ 
\hspace{4cm}($2^{i+1} = 2 * 2^i$)\\
Dies liefert eine rekursive Implementierung von $REACH(s,s',i)$ \\
\hspace{4cm}($\exists \check{s} \leadsto \text { for } \check{s} \in \text{ globale Konfigurationen}$)


Der Rekursionsstack hat Tiefe $S$ (Toplevel-Aufruf $REACH(s_{INI}, s_{ACC}, S)$).
\\
Jeder Activationrecord hat Größe $\bigO(S)$ genauer gesagt $2 S$ für die beiden Parameter $s, s', \log(S)$ für die Parameter $i$. Wenn gewünscht noch ein weiteres $S$ für die for-Schleife.


Die Gesamtgröße des Stacks ist beschränkt durch $S * \bigO(S) = \bigO(S^2)$.


Historisch wurde zunächst gezeigt, dass STCONN in $DSPACE(\log(n)^2)$ liegt. Der Satz von Savitch kann auch hieraus abgeleitet werden.






% chapter platzkomplexitaet (end)
