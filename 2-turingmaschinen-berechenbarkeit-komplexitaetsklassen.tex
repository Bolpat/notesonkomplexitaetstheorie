%!TEX root = 0-main.tex

% Author: Philipp Moers <soziflip@gmail.com> 




\chapter{Turingmaschinen, Berechenbarkeit und Komplexität} % (fold)
\label{cha:turingmaschinen_berechenbarkeit_und_komplexitaet}


\datum{13.10.15}

\section{Turingmaschinen}


\begin{definition}
        
    Eine \definiere{Turingmaschine} $T$ mit $k$ Bändern ist ein 5-Tupel
    $$ T = (Q, \Sigma, I, q_0, F) $$
    \begin{itemize}
        \item $Q$ ist eine endliche Menge von Zuständen
        \item $\Sigma$ ist eine endliche Menge von Bandsymbolen, $\square \in \Sigma$
        \item $I$ ist eine Menge von Quintupeln der Form $(q, s, s', m, q')$ mit $q, q' \in Q$ und $s, s' \in \Sigma^k$ und $m \in \{ L, R, S \}^k$ 
        \item $q_0 \in Q$ Startzustand
        \item $F \subseteq Q$ Endzustände
    \end{itemize}

    $\square$ ist das Leerzeichen oder \definiere{Blanksymbol}.

    $T$ heißt \definiere{deterministisch} genau dann, wenn für jedes $q \in Q$ und $s \in \Sigma^k$ genau ein Quintupel der Form $(q, s, \_, \_, \_) \in I$ existiert. Sonst heißt $T$ \definiere{nichtdeterministisch}.

    Eine Turingmaschine heißt \definiere{Akzeptormaschine} genau dann, wenn zwei Zustände $q_A, q_R \in F$ speziell markiert sind. $q_A$ signalisiert Akzeptanz, $q_R$ signalisiert Verwerfen der Eingabe.

    Eine Turingmaschine heißt \definiere{Transducermaschine} genau dann, wenn ein zusätzliches Band ausgezeichnet ist (das Ausgabeband).

\end{definition}



\begin{beispiel}
    Akzeptormaschine $T$ für Sprache $L = \{ 0^n 1^n | n \geq 0 \}$ wobei $\Sigma= \{0,1\}, Q = \{q_0, \dots q_4 \}$

    $T$ wird deterministisch sein. $T = (Q, \Sigma, I, q_0, F), q_A = q_1, q_R = q_2, F = \{q_1, q_2\}, k=2$

    \vspace{2pt}
    \begin{tabular}{|c|c|c|c|c|c|c|c|}\hline
    \rowcolor{grau} $q$   & $s_1$     & $s_2$     & $s_1'$    & $s_2'$    & $m_1$ & $m_2$ & $q'$    \\\hline
                    $q_0$ & $\square$ & $\square$ & $\square$ & $\square$ & $S$   & $S$   & $q_1$   \\\hline
                    $q_0$ & $0$       & $\square$ & $0$       & $0$       & $R$   & $R$   & $q_3$   \\\hline
                    $q_0$ & $1$       & $\square$ & $1$       & $\square$ & $S$   & $S$   & $q_2$   \\\hline
                    $q_3$ & $\square$ & $\square$ & $\_$      & $\_$      & $\_$  & $\_$  & $q_2$   \\\hline
                    $q_3$ & $0$       & $\square$ & $0$       & $0$       & $R$   & $R$   & $q_3$   \\\hline
                    $q_3$ & $1$       & $\square$ & $1$       & $\square$ & $S$   & $L$   & $q_4$   \\\hline
                    $q_4$ & $0$       & $0$       & $\_$      & $\_$      & $\_$  & $\_$  & $q_2$   \\\hline
                    $q_4$ & $1$       & $0$       & $1$       & $0$       & $R$   & $L$   & $q_4$   \\\hline
                    $q_4$ & $0$       & $\square$ & $\square$ & $\square$ & $S$   & $S$   & $q_1$   \\\hline
                    $\_$  & $\_$      & $\_$      & $\_$      & $\_$      & $\_$  & $\_$  & $q_2$   \\\hline
    \end{tabular}

\end{beispiel}


Die \definiere{globale Konfiguration} (oder der \definiere{Zustand}) einer Turingmaschine beinhaltet die Beschriftung aller Bänder, den internen Zustand ($\in Q$) und die Positionen aller $k$ Lese-/Schreibköpfe. Globale Konfigurationen können als endliche Wörter über einem geeigneten Alphabet (z.B. $\{0,1\}$) codiert werden.


Eine Turingmaschine \definiere{akzeptiert} eine Eingabe genau dann, wenn eine Berechnungsfolge ausgehend von dieser Eingabe existiert und in einem Zustand aus $F$ endet.

Eine Turingmaschine \definiere{akzeptiert} eine Sprache $L \subseteq \left( \Sigma \setminus \{\square\} \right)^\ast$ falls gilt: 
$$ \text{Die Turingmaschine akzeptiert } w \Leftrightarrow w \in L $$

Eine Turingmaschine \definiere{entscheidet} eine Sprache $L \subseteq \left( \Sigma \setminus \{\square\} \right)^\ast$ genau dann, wenn sie sie akzeptiert und eine/die Berechnung in $q_A$ endet.




% chapter turingmaschinen_berechenbarkeit_und_komplexitaet (end)