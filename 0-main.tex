% Author: Philipp Moers <soziflip@gmail.com> 

\documentclass[12pt, oneside, a4paper, numbers=enddot, abstracton, parskip=full]{scrreprt}

% Author: Philipp Moers <soziflip+latex@gmail.com>


\usepackage[utf8]{inputenc}
\usepackage[T1]{fontenc}

\usepackage[ngerman, english]{babel}
% document: \selectlanguage{ngerman}


\usepackage{hyphenat}

\usepackage{xifthen}

% Häkchen mit \checked
\usepackage{wasysym} 

% URLs
% \usepackage[hyphens]{url}
% hyperref loads url internally!
\PassOptionsToPackage{hyphens}{url}
\usepackage{hyperref}

% Uhrzeit mit \currenttime
\usepackage{datetime}

\newcommand{\todo}[1]   {\textcolor{red}{TODO: #1}} 


\usepackage{moreverb}



%%%%%%%%%%%%%%% Font %%%%%%%%%%%%%%%


% \renewcommand*{\familydefault}{\rmdefault}
\renewcommand*{\familydefault}{\sfdefault}
% \renewcommand*{\familydefault}{\ttdefault}


% Achtung, eigener Font im Mathematik Abschnitt!



%%%%%%%%%%%%%%% Unicode characters %%%%%%%%%%%%%%%

\usepackage{newunicodechar}
\usepackage{marvosym}

% \newunicodechar{∈}{\in}
\newunicodechar{∈}{$\in$}
\newunicodechar{€}{$\EURdig$}





%%%%%%%%%%%%%%% Farben %%%%%%%%%%%%%%%

\usepackage[usenames,dvipsnames,svgnames,table]{xcolor}
\definecolor{grau}{gray}{.90}
% \definecolor{orange}{RGB}{255,127,0}
\definecolor{myred}{HTML}{E78356}
\definecolor{myorange}{HTML}{FCD078}
\definecolor{mylightblue}{HTML}{A7CED1}
\definecolor{mylightgreen}{HTML}{9DE66E}
\definecolor{mylightpurple}{HTML}{FF9BF6}
\definecolor{mark}{HTML}{FFDDCC}


%%%%%%%%%%%%%%%	Tabellen %%%%%%%%%%%%%%%

\usepackage{tabularx}
\usepackage{colortbl}
\usepackage{multirow}
\newcolumntype{C}[1]{>{\centering\arraybackslash}m{#1}}


%%%%%%%%%%%%%%%	Layout etc. %%%%%%%%%%%%%%%

% \usepackage{enumerate}
% \usepackage{paralist} % compactenum

%%% IN JEDEM ÜBUNGSBLATT %%%
% \usepackage{sectsty}
% % \allsectionsfont{\large\underline} % überschriftendesign
% \sectionfont{\large\underline} % überschriftendesign
% \subsectionfont{\normalsize}


\usepackage{caption}
\captionsetup{margin=10pt,font=small,labelfont=bf}

\usepackage{courier}

\usepackage{pgf} \pgfmathsetmacro{\dotradius}{0.05}

\usepackage[a4paper]{geometry}

\usepackage{rotating}

\usepackage{setspace} \doublespacing

\renewcommand{\baselinestretch}{1.25}\normalsize
\renewcommand{\arraystretch}{1.0}


%%%%%%%%%%%%%%%	Übungsblatt Header %%%%%%%%%%%%%%%

% \newcommand{\header}[3]{\setlength{\parindent}{0pt}#1\hfill\today\\[12pt]\begin{huge}\textbf{#2}#3\end{huge}\\\rule{1\textwidth}{0.3pt}\\[9pt]}
\newcommand{\header}[3]
{\setlength{\parindent}{0pt}
\hfill\today\\
#1\\[18pt]
\begin{large}\textbf{#2}\end{large}\\[9pt]
\begin{huge}\textbf{#3}\end{huge}\\
\rule{1\textwidth}{0.3pt}\\[6pt]}



%%%%%%%%%%%%%%% Plots %%%%%%%%%%%%%%%

% \usepackage{gnuplottex}




%%%%%%%%%%%%%%%	Zeichnungen %%%%%%%%%%%%%%%

\usepackage{tikz}
\usetikzlibrary{decorations.markings,arrows} 
\usetikzlibrary{shapes,positioning,shadows,arrows,automata}
\tikzstyle{mystate} = [state, minimum size = 1.5cm, node distance = 2.5cm]
\tikzstyle{arrowlink} = [decoration={markings,mark=at position 0.1 with \arrow{triangle 60}}, postaction=decorate, color=black] 
\tikzstyle{nippel} = [circle, fill = myorange, draw, font=\sffamily\bfseries, minimum size=1.5cm]
\tikzstyle{bucket} = [rectangle, draw, font=\sffamily\bfseries, auto, thick, node distance = 1cm, minimum size=1cm]
\tikzstyle{decision} = [diamond, draw, fill=mark, text width=4.5em, text badly centered, node distance=3cm, inner sep=0pt]
\tikzstyle{block} = [rectangle, draw, fill=mark, text width=5em, text centered, rounded corners, minimum height=4em]
\tikzstyle{line} = [draw, -latex'] 
\tikzstyle{bplus}=[rectangle split, rectangle split horizontal,rectangle split ignore empty parts,draw]


% entity relationship diagram
% \usepackage{tikz-er2}
% \tikzstyle{every entity} = [top color=white, bottom color=myorange!30, draw=myorange!50!black!100, drop shadow]
% % \tikzstyle{every weak entity} = [drop shadow={shadow xshift=.7ex, shadow yshift=-.7ex}]
% \tikzstyle{every attribute} = [top color=white, bottom color=mylightblue!20, draw=mylightblue, node distance=1cm, drop shadow]
% \tikzstyle{every relationship} = [top color=white, bottom color=myred!20, draw=myred!50!black!100, drop shadow]
% % \tikzstyle{every isa} = [top color=white, bottom color=green!20, draw=green!50!black!100, drop shadow]


%%%%%%%%%%%%%%% Programmcode %%%%%%%%%%%%%%%

\usepackage{listings}

% \usepackage{listingsutf8} % kannste vergessen, chars müssen in ein byte passen

% replace special characters
\lstset{
  literate={ö}{{\"o}}1
           {Ö}{{\"O}}1
           {ä}{{\"a}}1
           {Ä}{{\"A}}1
           {ü}{{\"u}}1
           {Ü}{{\"U}}1
           {ß}{{\ss}}1
}

\usepackage{framed}
\colorlet{shadecolor}{grau}
\newcommand{\codeline}[1]			{\colorbox{grau}{\texttt{#1}}}
% \newcommand{\codebox}[1]			{\begin{shaded}\texttt{#1}\end{shaded}}
% \newcommand{\codebox}[1]			{\begin{lstlisting}{\texttt{#1}}\end{lstlisting}}
\lstnewenvironment{codebox}[1][] 	{\lstset{		language = #1,
													backgroundcolor = \color{grau},
                                                    % basicstyle = \ttfamily,
                                                    basicstyle = \scriptsize,
                                                    % basicstyle = \tiny,
                                                    numbers = left, % none
                                                    numberstyle = \tiny,
                                                    breaklines = true,
                                                    breakatwhitespace = true,
                                                    escapeinside = {@}{@},
                                                    frame = single,
                                                    tabsize = 2
}}{}
\newcommand{\codefile}[3] 		{\lstinputlisting[	language = #1,
                                                    % extendedchars = true,
                                                    % inputencoding = utf8/utf8,
													backgroundcolor = \color{grau},
                                                    basicstyle = \ttfamily,
                                                    numbers = left, % none
                                                    numberstyle = \tiny,
                                                    breaklines = true,
                                                    breakatwhitespace = true,
                                                    frame = single,
                                                    tabsize = 2,
													% firstline=42
													% lastline=50
													#2]
{#3}}


% JAVASCRIPT SUPPORT
\lstdefinelanguage{JavaScript}{
  keywords={typeof, new, true, false, catch, function, return, null, catch, switch, var, if, in, while, do, else, case, break},
  keywordstyle=\color{blue}\bfseries,
  ndkeywords={class, export, boolean, throw, implements, import, this},
  ndkeywordstyle=\color{darkgray}\bfseries,
  identifierstyle=\color{black},
  sensitive=false,
  comment=[l]{//},
  morecomment=[s]{/*}{*/},
  commentstyle=\color{purple}\ttfamily,
  stringstyle=\color{red}\ttfamily,
  morestring=[b]',
  morestring=[b]"
}
\lstset{
   language=JavaScript,
   backgroundcolor=\color{lightgray},
   extendedchars=true,
   basicstyle=\footnotesize\ttfamily,
   showstringspaces=false,
   showspaces=false,
   numbers=left,
   numberstyle=\footnotesize,
   numbersep=9pt,
   tabsize=2,
   breaklines=true,
   showtabs=false,
   captionpos=b
}






%%%%%%%%%%%%%%%	Mathematik etc. %%%%%%%%%%%%%%%

% \usepackage{MnSymbol}

% conflict with wasysym package
\let\iint\relax
\let\iiint\relax

% \usepackage{amsmath} % not needed with mathtools
\usepackage{amssymb}
% \usepackage{complexity} % not compatible with something

\usepackage{mathtools}


\usepackage{mathcomp}



% cool mathmode font
\usepackage{eulervm}
% ... with global font
\usepackage{mathpazo}


% --- Einheiten: ---
% (Syntax: \SI{wert}{einheit})
% \usepackage{siunitx}
% \sisetup
% {
%  	alsoload=binary,    % Binäre Einheiten (\bit, \byte)
%   % alsoload=synchem,   % Chemische Einheiten (\mmHg, \molar, \torr, ...)
%   % alsoload=astro,     % Astronomische Einheiten (\parsec, \lightyear)
%   % alsoload=hep,       % Einheiten der Hochenergie-Physik (\clight, \eVperc)
%   % alsoload=geophys,   % Einheiten der Geophysik (\gon)
%   % alsoload=chemeng    % Einheiten der chem. Verfahrenstechnik (\gmol, \kgmol, ...)
% 	unitsep=thin, valuesep=space
% }

% \newcommand{\lor}				{\ensuremath{\vee}}
% \newcommand{\land}			{\ensuremath{\wedge}}
\newcommand{\lxor}				{\ensuremath{\oplus}}

\newcommand{\setunion} 			{\ensuremath{\cup}}
\newcommand{\setintersection} 	{\ensuremath{\cap}}
 
\newcommand{\R}					{\ensuremath{\mathbb{R}}}
\newcommand{\N}					{\ensuremath{\mathbb{N}}}
\newcommand{\Z}					{\ensuremath{\mathbb{Z}}}

\newcommand{\bigO}				{\ensuremath{\mathcal{O}}} 			% big-O notation/symbol

\newcommand{\ComplexityClassP}         {\ensuremath{P}}      % complexity class
\newcommand{\ComplexityClassE}         {\ensuremath{E}}      % complexity class
\newcommand{\ComplexityClassEXP}       {\ensuremath{EXP}}      % complexity class
\newcommand{\ComplexityClassNP}        {\ensuremath{NP}}      % complexity class
\newcommand{\ComplexityClassNE}         {\ensuremath{NE}}      % complexity class
\newcommand{\ComplexityClassNEXP}       {\ensuremath{NEXP}}      % complexity class

%!TEX root = 0-main.tex

% Author: Philipp Moers <soziflip@gmail.com> 



\newcommand{\datum}[1]
{
    \begin{center}
    \textcolor[HTML]{66CC66}
    {
        \rule{1\textwidth}{0.3pt}\\[6pt]
        Vorlesung vom #1
        \rule{1\textwidth}{0.3pt}\\[6pt]
    }
    \end{center}
}


% \newenvironment{definition}[1][]
\newenvironment{definition}
{
    % \textbf{Definition: #1}

    \underline{\textbf{Definition}}

    % \begin{addmargin}[3][0]
    % \newgeometry{left=3cm,bottom=0.1cm}
    % \begin{quote}
}
{
    % \end{quote}
    % \restoregeometry
    % \end{addmargin}
}


\newcommand{\definiere}[1]
{
    \textbf{#1}
}



\newenvironment{beispiel}
{
    \begin{quote}
    \underline{\textbf{Beispiel}}
    
}
{
    \end{quote}
}

\begin{document}
\selectlanguage{ngerman}

\begin{titlepage}
    \begin{center}
        \Large{Ludwig-Maximilians-Universität München}\\[1cm]
        \large{\scshape{WS 2015/2016}}\\
        \large{\scshape{Martin Hofmann, Ulrich Schöpp}}\\[3cm]
        \Huge{\textbf{Komplexitätstheorie}}\\[5cm]
        \large{Vorlesungsmitschrieb von}\\[1cm]
        \large{Philipp Moers \\ 
        <p.moers@campus.lmu.de>\\
        <soziflip@gmail.com>}\\[2cm]
        \vfill
        \footnotesize{Last updated: \today, \currenttime}
    \end{center}
\end{titlepage}

\begin{abstract}

    Die Komplexitätstheorie beschäftigt sich mit der Klassifikation von Algorithmen und Berechnungsproblemen nach ihrem Ressourcenverbrauch, z.B. Rechenzeit oder benötigtem Speicherplatz. Probleme mit gleichartigem Ressourcenverbrauch werden zu Komplexitätsklassen zusammengefasst. Die bekanntesten Komplexitätsklassen sind sicherlich P und NP, die die in polynomieller Zeit deterministisch bzw. nicht-deterministisch lösbaren Probleme umfassen.

    P und NP sind jedoch nur zwei Beispiele von Komplexitätsklassen. Andere Klassen ergeben sich etwa bei der Untersuchung der effizienten Parallelisierbarkeit von Problemen, der Lösbarkeit durch zufallsgesteuerte oder interaktive Algorithmen, der approximativen Lösung von Problemen, um nur einige Beispiele zu nennen.

    
    \begin{center}
    \textbf{Anmerkung}
    \end{center}

    Dies ist ein inoffizieller Vorlesungsmitschrieb. Als solcher erhebt er keinen Anspruch auf (NP-) Vollständigkeit oder Korrektheit. Nutzung, Anmerkungen und Korrekturen sind jedoch durchaus erwünscht! 

    Vorlesungs-Website: \url{http://www.tcs.ifi.lmu.de/lehre/ws-2015-16/kompl}
    
\end{abstract}


\tableofcontents


\newpage

%!TEX root = 0-main.tex

% Author: Philipp Moers <soziflip@gmail.com> 



\chapter{Einführung} % (fold)
\label{cha:einfuhrung}



\section{Motivation}

Theoretische Informatik, Berechenbarkeit und insbesondere Komplexitätstheorie ist \emph{der} Informatiker-Shit schlechthin. Let's do it!



\section{Literatur}

Die Vorlesung basiert hauptsächlich auf folgendem Buch:
\begin{itemize}
    \item Bovet, Crescenzi. Introduction to the Theory of Complexity. Prentice Hall. New York. 1994.
\end{itemize}

Weiterhin ist folgende Literatur gegeben:
\begin{itemize}
    \item C. Papadimitriou. Computational Complexity. Addison-Wesley. Reading. 1995.
    \item I. Wegener. Komplexitätstheorie: Grenzen der Effizienz von Algorithmen. Springer. 2003.
    \item S. Arora und B. Barak. Complexity Theory: A Modern Approach. 
\end{itemize}

Zur Motivation:
\begin{itemize}
    \item Heribert Vollmer. Was leistet die Komplexitätstheorie für die Praxis? Informatik Spektrum 22 Heft 5, 1999.
    \item Stephen Cook: The Importance of the P versus NP Question. Journal of the ACM (Vol. 50 No. 1)
\end{itemize}







% chapter einfuhrung (end)

%!TEX root = 0-main.tex

% Author: Philipp Moers <soziflip@gmail.com> 




\chapter{Turingmaschinen, Berechenbarkeit und Komplexität} % (fold)
\label{cha:turingmaschinen_berechenbarkeit_und_komplexitaet}


\datum{13.10.15}

\section{Turingmaschinen}


\begin{definition}
        
    Eine \definiere{Turingmaschine} $T$ mit $k$ Bändern ist ein 5-Tupel
    $$ T = (Q, \Sigma, I, q_0, F) $$
    \begin{itemize}
        \item $Q$ ist eine endliche Menge von Zuständen
        \item $\Sigma$ ist eine endliche Menge von Bandsymbolen, $\square \in \Sigma$
        \item $I$ ist eine Menge von Quintupeln der Form $(q, s, s', m, q')$ mit $q, q' \in Q$ und $s, s' \in \Sigma^k$ und $m \in \{ L, R, S \}^k$ 
        \item $q_0 \in Q$ Startzustand
        \item $F \subseteq Q$ Endzustände
    \end{itemize}

    $\square$ ist das Leerzeichen oder \definiere{Blanksymbol}.

    $T$ heißt \definiere{deterministisch} genau dann, wenn für jedes $q \in Q$ und $s \in \Sigma^k$ genau ein Quintupel der Form $(q, s, \_, \_, \_) \in I$ existiert. Sonst heißt $T$ \definiere{nichtdeterministisch}.

    Eine Turingmaschine heißt \definiere{Akzeptormaschine} genau dann, wenn zwei Zustände $q_A, q_R \in F$ speziell markiert sind. $q_A$ signalisiert Akzeptanz, $q_R$ signalisiert Verwerfen der Eingabe.

    Eine Turingmaschine heißt \definiere{Transducermaschine} genau dann, wenn ein zusätzliches Band ausgezeichnet ist (das Ausgabeband).

\end{definition}



\begin{beispiel}
    Akzeptormaschine $T$ für Sprache $L = \{ 0^n 1^n | n \geq 0 \}$ wobei $\Sigma= \{0,1\}, Q = \{q_0, \dots q_4 \}$

    $T$ wird deterministisch sein. $T = (Q, \Sigma, I, q_0, F), q_A = q_1, q_R = q_2, F = \{q_1, q_2\}, k=2$

    \vspace{2pt}
    \begin{tabular}{|c|c|c|c|c|c|c|c|}\hline
    \rowcolor{grau} $q$   & $s_1$     & $s_2$     & $s_1'$    & $s_2'$    & $m_1$ & $m_2$ & $q'$    \\\hline
                    $q_0$ & $\square$ & $\square$ & $\square$ & $\square$ & $S$   & $S$   & $q_1$   \\\hline
                    $q_0$ & $0$       & $\square$ & $0$       & $0$       & $R$   & $R$   & $q_3$   \\\hline
                    $q_0$ & $1$       & $\square$ & $1$       & $\square$ & $S$   & $S$   & $q_2$   \\\hline
                    $q_3$ & $\square$ & $\square$ & $\_$      & $\_$      & $\_$  & $\_$  & $q_2$   \\\hline
                    $q_3$ & $0$       & $\square$ & $0$       & $0$       & $R$   & $R$   & $q_3$   \\\hline
                    $q_3$ & $1$       & $\square$ & $1$       & $\square$ & $S$   & $L$   & $q_4$   \\\hline
                    $q_4$ & $0$       & $0$       & $\_$      & $\_$      & $\_$  & $\_$  & $q_2$   \\\hline
                    $q_4$ & $1$       & $0$       & $1$       & $0$       & $R$   & $L$   & $q_4$   \\\hline
                    $q_4$ & $0$       & $\square$ & $\square$ & $\square$ & $S$   & $S$   & $q_1$   \\\hline
                    $\_$  & $\_$      & $\_$      & $\_$      & $\_$      & $\_$  & $\_$  & $q_2$   \\\hline
    \end{tabular}

\end{beispiel}


Die \definiere{globale Konfiguration} (oder der \definiere{Zustand}) einer Turingmaschine beinhaltet die Beschriftung aller Bänder, den internen Zustand ($\in Q$) und die Positionen aller $k$ Lese-/Schreibköpfe. Globale Konfigurationen können als endliche Wörter über einem geeigneten Alphabet (z.B. $\{0,1\}$) codiert werden.


Eine Turingmaschine \definiere{akzeptiert} eine Eingabe genau dann, wenn eine Berechnungsfolge ausgehend von dieser Eingabe existiert und in einem Zustand aus $F$ endet.

Eine Turingmaschine \definiere{akzeptiert} eine Sprache $L \subseteq \left( \Sigma \setminus \{\square\} \right)^\ast$ falls gilt: 
$$ \text{Die Turingmaschine akzeptiert } w \Leftrightarrow w \in L $$

Eine Turingmaschine \definiere{entscheidet} eine Sprache $L \subseteq \left( \Sigma \setminus \{\square\} \right)^\ast$ genau dann, wenn sie sie akzeptiert und eine/die Berechnung in $q_A$ endet.


\begin{definition}

\definiere{Nichtdet Zeitkomplexitaet}
Seit T eine nichtdet. TM, Fuer x \in \Sigma ist NTime_T(x)

\begin{enumerate}
\item definiert g.d.w. alles Berechnungen von T auf x halten
\item Falls definiert  und x \inL(T) (d.h. es gibt eine akyeptierende Ber. von T auf x)
  definiert als Laenge der kuerzesten akzeptierenden Berechnungen von T auf X.
\item Falls ueberhaupt definiert x \notin L(T) so ist NTIME_{(x)} die Laenge der kuerzesten Berechnung
\end{enumerate}

\end{definition}

\begin{definition}
  Nichtdeterministische Komplexitaetsklassen NTIME(f(n)) = { L | esox T L(T)=L und NTIME_T(x) = O(f(|x|))}

  Es gibt einen nichtdeterministischen Zeithierachiesatz.


  P=U DTIME(n^k), NP = U NTIME(n^k)
  k>=1                k>=1

  E = U DTIME(2^kn), NE= U NTIME(2^KN)
  k>=1               k>=1

  EXP = DTIME(2^n^k), NEXP= U NTIME(2^N^k)
  k>=1                k>=1


  Nichtdeterminismus kann durch exhaustive Suche deterministsch simuliert werden.
  z.B. NP \subseteq EXP


  Allgemein:
  NTIME(f(n)) \subseteq DTIME(2^O(f(n))
  ........\ --- Zeitkonstruierbar

  Charakteriserung von NP durch ``polynoielle Verizibierbakeit''

  L\subseteq \sigma * . L \in PV \Leftrightarrow

  eine Spraceh L' \in P existiert so dass gilt

  x\inL \Leftrightarrow  esex z ``Loesung'' mit |h|<=L'
  wobei p ein Polynom ist.

\end{definition}

\begin{satz}
  NP = PV
  Beweisskizze:
  ``S'' L \in NP . Sei T eine NTM fuer L mit Laufzeit p(n).

  L' = {(x,y) | y codiert eine akyeptierende Berechnung von T auf x}

  \begin{enumerate}
  \item x \in L \Leftrightarrow  \E y |y| <= (p(|x|))^2 . (x,y,) \in L'

  \item L' \in P

  \end{enumerate}


``\supseteq'' geg. L, L'\in P . Eine nicht det Turingmaschine T fuer L raet zunaechst y und prueft dann (x,y) \in L'
  
\end{satz}

Exp im Gegensartz zu NP bzw. PV umfasst auch Probleme mit exp. groesen Loesungen bzw solchen wo die Verifikation einer Loesung einen exponentiellen Aufwand macht.


% chapter turingmaschinen_berechenbarkeit_und_komplexitaet (end) 

%!TEX root = 0-main.tex

% Author: Philipp Moers <soziflip@gmail.com> 




\chapter{NP und P} % (fold)
\label{cha:np_und_p}
        

\section{Padding}

\begin{definition}
    Sei $L \subseteq \Sigma^\ast$ eine Sprache.

    % (Blase: 2^?  ist die Paddingfunktion geht auch für andere P.fkt.)
    % TODO verstehen, anpassen, einkommentieren

    $$ padd(L) = \{1^l O x | x \in L, l = 2^{|x|} \} $$
\end{definition}


\begin{satz}
    
    Es gilt: 
    $ L \in DTIME(f(s^n)) $, dann ist
    $$ padd(L) \in DTIME(f(n)) $$
    Blase $f$ zeitkonstant und insbesondere $f(n) \geq n$

\end{satz}

\begin{beweis}
    
    Sei $T$ eine deterministische Turingmaschine für $L$ und
    % TODO \leq \in ?
    $DTIME_T(x) \leq \in x f(2^n)$

    Die folgende Maschine $T'$ entscheided $padd(L)$:

    Gegeben Eingabe $y$, schreibe $y + 1^l O x$ und prüfe ob $l = 2^{|x|}$
    geht in Zeit $\bigO(|y|)$

    Aufwand: 
    $c f(2^{|x|}) \leq c * f(|y|)$

    Gesamtaufwand:
    $\leq c * f(|y|) + |y|
    = \bigO(f(|y|))$

    falls $f(n) \geq n$ (zeitkonstruierbar)
\end{beweis}




\begin{satz}
    Umgekehrt gilt auch:

    Wenn $padd(L) \in DTIME(f(n))$
    dann $L \in DTIME(f(s^{n+1}))$
\end{satz}

\begin{beweis}
    Sei $T$ eine Turingmaschine für $padd(L)$
    mit $DTIME_T(y) \leq c f(|y|)$

    Wir bauen eine Maschine für $L$:
    Gegeben Eingabe $x$, bilde $y = 1^{2^{|x|}} O x $.

    Aufwand: $\bigO(2^{|x|})$
    Setze $T$ auf $y$ an.
    Aufwand: $c * f(|y|) = c * f(2^{|x|} + |x| + 1)$

    Gesamtaufwand:
    $\bigO(f(2^{|x|+1}))$
\end{beweis}




\section{Was wenn P = NP}

\begin{satz}
    Folgerung:
    $$\ComplexityClassP = \ComplexityClassNP \Rightarrow \ComplexityClassE = \ComplexityClassNE$$
\end{satz}
\begin{beweis}
    Sei $P = NP$
    und $L \in NE$ und $T$ eine Maschine mit Aufwand $n^{kk}$
    wobei $k$ fest, $n$ Länge der Eingabe.
    
    $L \in NTIME(n^{nk}) = NTIME((2^n)^k)$
    Also $padd(L) \in NTIME(2^k)$
    also $padd(L) \in  NP$ und nach Annahme $padd(L) \in P$

    Also $padd(L) \in DTIME(n^{k'})$
    also $L \in DTIME((2^{n+1})^{k'})
    = DTIME(2^{k'n + k'}) = DTIME(2^{k'n}) \subseteq E$
\end{beweis}



Slogan: Gleichheit von Komplexitätsklassen vererbt sich nach oben.

Mit anderen Paddingfunktion zeigt man ebenso:

$$ \ComplexityClassP = \ComplexityClassNP \Rightarrow \ComplexityClassEXP = \ComplexityClassNEXP $$
$$ \ComplexityClassE = \ComplexityClassNE \Rightarrow \ComplexityClassEXP = \ComplexityClassNEXP $$

Kontrapositiv ausgedrückt:

$$ \ComplexityClassE \neq \ComplexityClassNE \Rightarrow \ComplexityClassP \neq \ComplexityClassNP $$ etc.


Es koennte sein, dass $\ComplexityClassP \neq \ComplexityClassNP$ aber doch $\ComplexityClassE = \ComplexityClassNE$.

Slogan: Trennung von Komplexitätklassen vererbt sich (durch Padding) von oben nach unten.




\section{Sogenannte Effizienz}


$\ComplexityClassP$ wird gemeinhin gleichgesetzt mit effizienter Lösbarkeit.
    
Wachstumsverhalten:

  $p$ Polynom. $\Rightarrow$ 
  $$ \exists c > 0:   p(2n) \leq c * p(n) $$

  Bei Verdopplung des Inputs wird der Output also ver-$c$-facht.

  Häufig hat stures Durchprobieren, auch Bruteforcing genannt,
  exponentiellen und eine echte algorithmische Lösung hat polynomiellen Aufwand.











\datum{09.11.15}


\section{Polynomialzeitreduktionen}



\begin{definition}

    $f : \Sigma^\ast \rightarrow \Sigma^\ast $
    beziehungweise für Binärkodierung
    $f : \N \rightarrow \N $

    ist in \definiere{$\ComplexityClassFP$} (eine \definiere{in Polynomialzeit berechenbare Funktion}) genau dann, wenn eine polynomialzeitbeschränkte (deterministische) Transducer-Maschine existiert, die $f$ berechnet.

\end{definition}


Gegeben Input $x \in \Sigma^\ast$, dann hält die Maschine nach $\leq p(|x|)$ Schritten mit Ergebnis $f(x)$, wobei $p$ ein Polynom ist.

\begin{beispiel}
    
    \textbf{Musterbeispiele:}

    \begin{itemize}
        \item Alle Polynome sind in $\ComplexityClassFP$, zum Beispiel $f(x) = x^3 + 10x^2 + x$
        \item Charakteristische Funktionen aller Probleme in $\ComplexityClassP$ sind in $\ComplexityClassFP$.
        \item Matrixmultiplikation ist in $\ComplexityClassFP$.
    \end{itemize}

    \textbf{Gegenbeispiele:}

    \begin{itemize}
        \item $f(x) = 2^x$ ist nicht in $\ComplexityClassFP$, denn $|2^x| = x + 1 \geq 2^{|x| + 1} +  1 = \Omega(2^{|x|})$
        \item Charakteristische Funktionen von Problemen in $\ComplexityClassEXP \setminus \ComplexityClassP$  sind nicht in $\ComplexityClassFP$.
    \end{itemize}

    \textbf{Wahrscheinliches Gegenbeispiel:}

    $f(n) = \text{größter Teiler von } n \text{ außer } n \text{ selbst.}$
    Algorithmus: Alle Zahlen von 1 bis $n$ durchlaufen und testen, immer merken, wenn größerer Teiler gefunden.\\
    Laufzeit: $\Omega(n) = \Omega(2^{|n|})$ (Länge der Eingabe statt Zahl selbst)

\end{beispiel}




\begin{satz}
    $\ComplexityClassFP$ ist unter Komposition (Hintereinanderausführung) abgeschlossen.\\
\end{satz}
\begin{beweis}
    
    Seien $ f: \Sigma^\ast \rightarrow \Sigma^\ast, g : \Sigma^\ast \rightarrow \Sigma^\ast \in \ComplexityClassFP $

    Wir definieren $h(x) = g(f(x))$

    Programm für $h$: $y = f(x);\ z = g(y);\ return\ z;$\\
    Gesamtaufwand: $O(p_f(|x|) + p_g(p_f(|x|))$, wobei $p_f$ und $p_g$ Polynome sind, die die Laufzeit für Algorithmen für $f$ bzw. $g$ beschränken.
\end{beweis}




\begin{definition}

    \definiere{Polynomielle Reduzierbarkeit}

    Seien $L_1, L_2a \subseteq \Sigma^\ast$.

    Wir sagen $L_1$ ist polynomiell auf $L_2$ reduzierbar (\definiere{$\ComplexityClassFP$-reduzierbar}) genau dann, wenn $f \in \ComplexityClassFP$ existiert, sodass $x \in L_1 \Leftrightarrow f(x) \in L_2$.

    Aus Algorithmus für $L_2$ erhält man einen für $L_1$, indem man $f(x)$ berechnet und prüft, ob das Ergebnis in $L_2$ liegt.

    In Zeichen: $L_1 \leq_p L_2$ oder $L_1 \leq L_2$, auch $f: L_1 \leq L_2$

\end{definition}


\begin{beispiel}
    
    3COL = $\{ G = (V,E)\ |\ G \text{ kann mit 3 Farben gefärbt werden, d.h. } \exists c: V \rightarrow \{r,g,b\} \text{ sodass } \forall (u, u') \in E: c(v) \neq c(u') \}$

    SAT = $\{ \phi\ |\ \phi \text{ aussagenlogische Formel die erfüllbar ist} \}$


    Behauptung: 3COL $\leq$ SAT



    % $f((V,E)) = (\bigwedge \bigvee         x_{v,y}  )      \land      \bigwedge   \bigwedge    \bigwedge      x_{v,c} \rightarrow \neg x_{v,c'}$
    %             v \in V    y \in r,g,b                               v \in V      c \in r,g,b    c' \in r,g,b
    %                                                                                               c \neq c'

    %             $ \land    \bigwedge        \bigwedge         x_{v,y} \rightarrow \neg x_{v', y}$
    %                        (v,v') \in E     y \in {r,g,b}


    $$f((V,E)) = \left(     \bigwedge_{v \in V} \bigvee_{y \in \{r,g,b\}}    x_{v,y}    \right)         \land     $$ 

    $$ \left(     \bigwedge_{v \in V} \bigwedge_{c \in \{r,g,b\}} \bigwedge_{c' \in \{r,g,b\}}      x_{v,c} \rightarrow \neg x_{v,c'} \right)       \land     $$

    $$ \left(     \bigwedge_{(v,v') \in E} \bigwedge_{y \in \{r,g,b\}}    x_{v,y} \rightarrow \neg x_{v', y} \right) $$
                                


    Offensichtlich ist $f \in \ComplexityClassFP$ und $G \in \text{ 3COL } \Leftrightarrow f(G) \in \text{ SAT }$,\\ also 3COL $\leq$ SAT.

\end{beispiel}



\begin{beispiel}
    
    KNFSAT := wie SAT, aber auf konjunktive Normalform eingeschränkt.

    Es gilt trivialerweise KNFSAT $\leq$ SAT, aber auch SAT $\leq$ KNFSAT (Durch Einführung von Abkürzungen von Teilformeln).

\end{beispiel}

\begin{beispiel}
    
    3SAT := KNFSAT eingeschränkt auf Klauseln mit 3 Literalen.

    Es gilt KNFSAT $\leq$ 3SAT.

    Man kennt keine Reduktion von 3SAT auf 2SAT.

\end{beispiel}


\begin{beispiel}
    
    NODE-COVER := $\{ G = (V,E), n\ |\ \exists U \subseteq V: |U| \leq n \text{ und } \forall (v,v') \in E: v \in U \lor v' \in U \}$

    Es gilt NODE-COVER $\leq$ KNFSAT.

    und - was schwieriger zu zeigen ist - KNFSAT $\leq$ NODE-COVER:

    Gegeben: KNF $\phi$ mit $m$ Variablen $x_1 \dots x_m$ und $k$ Klauseln $C_1 \dots C_k$ wobei $C_j = l_{j,1} \lor \dots \lor l_{j,k}$\\
    Falls 3SAT, so sind alle $k_j = 3$.\\
    Die $l_{j,i}$ sind Literale, d.h. negierte oder nicht-negierte Variablen.

    Wir konstruieren Graphen $G = (V,E)$ wie folgt:
    \begin{itemize}
        \item Für jede Variable $x_t$ zwei Knoten $x_t, \neg x_t$
        \item Für jede Klausel $C_j$ $k_j$ Knoten $(l_{j,1}) \dots (l_{j,k})$,\\ Insgesamt $2m \Sigma_{j=1}^k k_j$ Knoten
        \item Kanten: 
        \begin{itemize}
            \item $(x_t, \neg x_t)$
            \item Vollständiger Graph für $l_{j,1} \text{ bis } l_{j,k}$
            \item $(x_t, l_{j,i})$ bzw. $(\neg x_t, l_{j,i})$, falls $l_{j,i} = x_t$ bzw. $l_{j,i} = \neg x_t$
        \end{itemize}
    \end{itemize}

\end{beispiel}





\section{NP-Härte und NP-Vollständigkeit}


\begin{definition}
    
    $L \subseteq \Sigma^\ast$ ist \definiere{$\ComplexityClassNP$-hart} ($\ComplexityClassNP$-schwer, $\ComplexityClassNP$-schwierig) genau dann, wenn 

    $$ \forall L' \in \ComplexityClassNP: L' \leq_p L $$

\end{definition}


\begin{satz}
    HALT (Halteproblem) ist NP-hart.
\end{satz}
\begin{beweis}
    Gegeben: $L' \in \ComplexityClassNP$. Baue deterministische Turingmaschine $M$, sodass $M(x)$ hält genau dann, wenn $x \in L'$ Brute-force Suche, Laufzeit exponentiell.\\
    $x \in L' \Leftrightarrow (M, x) \in \text{ HALT } $\\
    also ist $f(x) = (M,x)$ eine Reduktion von $L'$ auf HALT $f: L' \leq \text{ HALT }$.
\end{beweis}



\begin{definition}

    $L \subseteq \Sigma^\ast$ ist \definiere{NP-vollständig} genau dann, wenn $L$ NP-hart ist und $L \in \ComplexityClassNP$.

\end{definition}


\begin{satz}
    HALT $\notin \ComplexityClassNP$.
\end{satz}



\begin{satz}
    \definiere{Satz von Cook}

    SAT ist NP-vollständig.
\end{satz}

\begin{beweis}
    
    SAT $\in \ComplexityClassNP$: trivial.

    Sei $L \in \ComplexityClassNP$ gegeben und o.B.d.A $M$ eine nichtdeterministische Turingmaschine für $L$ mit einem Band $M = (\Sigma, Q, q_0, F, I)$ und $p$ ein Polynom, das die Laufzeit von $M$ beschränkt.
    Gegeben weiterhin $x = x_1 \dots x_n$ Input.

    Gesucht: aussagenlogische Formel $\phi = f(x)$, sodass $\phi$ erfüllbar ist genau dann, wenn $M$ akzeptiert $x$. $q$ muss aus $x$ in polynomieller Zeit berechenbar sein, d.h. $f \in \ComplexityClassFP$.

    $M$ akzeptiert $x$ genau dann, wenn eine akzeptierende Berechnung von $M$ auf $x$ existiert. Solch eine Berechnung hat höchstens $p(n)$ Schritte und o.B.d.A genau $p(n)$ Schritte. Die Bandbeschriftung zu jedem dieser $p(n)$ Schritte besteht aus höchstens $p(n)$ Symbolen und o.B.d.A genau $p(n)$ Symbolen.

    Die Formel $\phi$ verwendet die Variablen 
    \begin{itemize}
        \item $Q_t^i$: Zur Zeit $t$ ist $M$ im Zustand $i$.
        \item $P_{s,t}^i$: Zur Zeit $t$ enthält Bandposition $s$ das $i.$ Symbol.
        \item $S_{s,t}$: Zur Zeit $t$ ist der Kopf in Position $s$.
    \end{itemize}

    $$ q = A \land B \land C \land D \land E \land F $$

    \textit{Details im Buch\dots}

\end{beweis}


3SAT, NODE-COVER sind auch NP-vollständig.

Allgemein gilt: $L$ NP-vollständig und $L' \in \ComplexityClassNP, L \leq L'$ 
so folgt $L'$ NP-vollständig.


\textbf{Anmerkung:}
$\leq$ ist transitiv, da $\ComplexityClassFP$ unter Komposition abgeschlossen ist.

\textbf{Anmerkung:}
3COL, TRAVELINGSALESMAN, SUBSETSUM etc. sind auch NP-vollständig.












\datum{12.11.15}


\section{Etwas zwischen P und NP}



\begin{satz}
    
    \definiere{Satz von Ladner}

    Falls $\ComplexityClassP \neq \ComplexityClassNP$, dann

    $$ \exists A \in \ComplexityClassNP \setminus \ComplexityClassP: A \text{ nicht NP-vollständig.} $$

    $A$ liegt also ``echt'' zwischen $\ComplexityClassP$ und NP-vollständig.

    
\end{satz}



\begin{definition}

    \definiere{Diagonalisierung}
    
    Um zu zeigen, dass eine Sprache $A$ nicht in einer Klasse $\mathcal{C}$ ist, beziehungweise um solch ein $A$ zu konstruieren, kann man eine effektive (FP) Aufzählung von Turingmaschine $(M_i)_i$ verwenden, sodass $\mathcal{C} = \{ L(M_i)\ |\ i \geq 0 \}$  und dann dafür sorgen, beziehungweise zeigen, dass $\forall i: A \neq L(M_i)$ beziehungweise $\forall i: A \triangle L(M_i) \neq 0$.\\
    Das heißt $\forall i \exists x: \left(x \in A \land x \notin L(M_i) \right) \lor \left( x \notin A \land x \in L(M_i) \right)$

\end{definition}


\begin{lemma}
    
    Es existiert eine FP-Funktion $i \mapsto M_i$, sodass $DTIME_{M_i}(x) \leq (|x| + 2)^2$ und $P = \{ L(M_i)\ |\ i \geq 0 \}$.

\end{lemma}

\begin{lemma}
    
    Es existiert eine FP-Funktion $i \mapsto f_i$ wobei $f_i$ eine Übersetzermaschine ist und $FP = \{ f_i\ |\ i \geq 0 \}$ und $DTIME_{f_i}(x) \leq (|x| + 2)^i$. Insbesondere $|f_i(x)| \leq (|x| + 2)^i$.

\end{lemma}

Es ist klar, dass $A \in \ComplexityClassNP$ aber $A \notin \ComplexityClassP$ und $A$ nicht NP-vollständig, wenn
\begin{itemize}
    \item $A \in NP$
    \item $\forall i \exists x: x \in A \triangle L(M_i)$
    \item $\forall i \exists x: x \in SAT \land f_i(x) \notin A \text{ oder } x \notin SAT \land f_i(x) \in A$
\end{itemize}

Das heißt $f_i$ ist keine Reduktion von $SAT$ auf $A$.





Wir konstruieren $A$ in der folgenden Form:

$$ A = \{  x \ |\   x \in SAT  \land   f(|x|) \text{ gerade.}  \} $$

$f$ wird sogleich rekursiv definiert derart, dass dieses $A$ die Bedingungen 1, 2 und 3 erfüllt.

Man sollte also versuchen sicherzustellen, dass 

\begin{itemize}
    \item 
    $f(n) in ZEit p(n)$ berechenbar für Polynom $p$ (Bedingung 1).

    \item 
    Für alle $i$ existiert $x$ mit \\
    $x \in SAT$ und $f(|x|)$ gerade und $x \notin L(M_i)$\\
    oder\\
    ($ x \notin SAT$ oder $f(|x|)$ ungerade ) und $x \in L(M_i)$\\

    \item 
    Für alle $i$ existiert $x$, sodass
    $x \in SAT$ und ($f(|f_i(x)|)$ ungerade oder $f_i(x) \notin SAT$\\
    oder\\
    $x \notin SAT$ und $f(|f_i(x)|)$ gerade und $f_i(x) \in SAT$\\

\end{itemize}


$f$ wird jetzt rekursiv definiert.\\
Wir schreiben $A_f = \{ x \ |\   x \in SAT  \land  f_(|x| \text{ gerade.}  \}$

\begin{equation*}
\begin{split}
f(n+1) = 
    IF    &\ \ (2 + \log \log n)^{f(n)} \geq \log n \\
    THEN  &\ \ f(n) \\
    ELIF  &\ \ \exists x: |x| \leq \log \log n \text{ und } x \in L(M_i) \land x \notin A_f \text{ oder } x \notin L(M_i) \land x \in A_f \\
    THEN  &\ \ f(n) + 1 \text{ ebe } f(n) \\
    ELIF  &\ \ \exists x: |x| \leq \log \log n \\
    THEN  &\ \ f(n) + 1
\end{split}
\end{equation*}

Um $f(n+1)$ zu berechnen wird rekursiv nur auf Werte $f(m)$ mit $m \leq n$ zugegriffen, also ist $f$ eine totale Funktion.

Es genügt, ein Polynom $p(n)$ zu finden, sodass in Zeit $p(n)$ der Wert $f(n+1)$ aus $f(0), f(1), \dots f(n)$ bestimmt werden kann.\\
Die Laufzeit für $f(n)$ ist nämlich dann $\bigO(\Sigma_{m < n} p(m)) = poly(n)$.


Klar ist, dass $A = A_f \neq L(M_i)$ falls $f(n) = 2i+1$ für ein $n$, denn dann war $f(n') = 2i$ für ein $n' < n$ und $f(n'+1) = 2i+1$ also die Suche in Fall 2 erfolgreich.

Ebenso ist $f_i$ keine Reduktion: $SAT \leq A_f$ falls $f(n) = (2i+1)+1$ für ein $n$.

Das heißt wir müssen zeigen, dass $f$ surjektiv ist, d.h. dass jeder Fall irgendwann erfolgreich abgeschlossen wird.

% Für festes $d$ wächst $(2 + \log n)^d$ langsamer als $n$, also wächst auch $ $ blabla

\textit{Details dazu auf der Website.}










\datum{16.11.15}


\section{Orakel-Turingmaschinen}


Orakel-Turingmaschinen als Mittel zu zeigen, dass Beweismethoden ``Diagonalisierung''\footnote{zum Beispiel benutzt für $\ComplexityClassNP \subseteq \ComplexityClassEXP$} und ``Simulation''\footnote{zum Beispiel benutzt für $\ComplexityClassP \subset \ComplexityClassEXP$ (Ladner)} nicht helfen, um $\ComplexityClassP = \ComplexityClassNP$ zu entscheiden.



\begin{definition}
    
    Eine \definiere{Orakel-Turingmaschine} $T$ hat ein zusätzliches Band (Orakelband) und drei zusätzliche Zustände $q_Q$ (Frage), $q_{yes}$, $q_{no}$ (Antwort).

    Ist $A \subseteq \Sigma^\ast$, dann definiert man Berechnungen $T^A(x)$ von $T$ auf $x$ mit Orakel $A$ wie folgt:
    \begin{itemize}
        \item Wie üblich mit der zusätzlichen Regel:\\
            Falls $T$ in $q_Q$ so wird $T$ in Zustand $q_{yes}$, $q_{no}$ versetzt und zwar in einem Schritt, je nach dem, ob die aktuelle Beschriftung $z \in \Sigma^\ast$ des Orakelbands (``Anfrage''/``Query'') in $A$ ist ($q_{yes}$) oder nicht ($q_{no}$).
    \end{itemize}
    Man schreibt $L^A(T)$ oder $L(T^A)$ für die von $T$ akzeptierte Sprache, falls Anfragen gemäß $A$ beantwortet werden.

\end{definition}




\begin{beispiel}
    
    Sei $STCONN = \{ (G, s, t) \ |\  \exists \text{ Pfad von } s \text{ nach } t \ \ \in G \} $

    Feststellen, ob ein Graph $G$ einen nichttrivialen Zyklus enthält. Zähle alle Paare $(u,v)$ auf und frage jeweils $(G, u, v)$ und $(G, v, u)$ ab.

    Damit ist eine Maschine $T$ beschrieben, sodass $CYCLE = L^{STCONN}(T)$.\\
    Die Laufzeit von $T$ ist $|G|^2$ (insbesondere polynomiell). Es ist also $CYCLE \in P^{STCONN}$. (Definition unten)\\
    Nachdem nun $STCONN \in \ComplexityClassP$ folgt $CYCLE \in \ComplexityClassP$.

\end{beispiel}


\begin{definition}

    Sei $A \subseteq \Sigma^\ast$.

    Man definiert $\ComplexityClassP^A$ als die Menge aller Sprachen $L$ sodass eine deterministische Turingmaschine $T$ existiert mit $L = L^A(T)$ und $DTIME(T^A(x)) \leq p(|x|)$ für ein Polynom $p$.

    Analog $\ComplexityClassNP^A$.
    
\end{definition}

\textbf{Beobachtung:}
$A \in \ComplexityClassP \Rightarrow \ComplexityClassP^A = \ComplexityClassP \land \ComplexityClassNP^A = \ComplexityClassNP$.

    
\begin{beispiel}
    
    $SAT \in \ComplexityClassP^{SAT}$

    $NODE-COVER \in \ComplexityClassP^{SAT}$

    $\ComplexityClassNP \in \ComplexityClassP^{SAT}$


    $ TAUT = \{ \phi \ |\  \phi \text{ allgemeingültig} \} \in \ComplexityClassP^{\ComplexityClassNP}$


    $ IMPL = \{ (\phi, \psi) \ |\  \phi \text{ allgemeingültig} \Rightarrow \psi \text{ allgemeingültig} \} \in \ComplexityClassP^{\ComplexityClassNP} $


    $ CIRCUIT-MIN = \{  (\text{Schaltkreis } C,  A) \ | \ 
            \text{Schaltkreis } C' \text{ der Größe } \leq k \text{ und } C \equiv C' \}  \in \ComplexityClassNP^{SAT}$

\end{beispiel}



Wr zeigen jetzt die folgenden zwei Sätze, aufgrund derer kein Beweis für $\ComplexityClassP = \ComplexityClassNP$ oder $\ComplexityClassP \neq \ComplexityClassNP$ existieren kann, welcher in Gegenwart von Orakeln auch funktioniert:

\begin{satz}
    
    $$ \exists A \in \Sigma^\ast: \ComplexityClassP^A = \ComplexityClassNP^A $$

\end{satz}

\begin{beweis}
    
    $A = \{ ( T, x, 0^k )  \ |\  
        \text{ deterministische Turingmaschine } T \text{ akzeptiert } x \text{ und benutt dabei } \leq k \text{ Bandzellen}  \}
    $

    $A$ ist offensichtlich entscheidbar.

    Sei $T$ eine nichtdeterministische Orakel-Turingmaschine und $p(n)$ ein Polynom, das die Laufzeit von $T$ beschränkt und somit auch die Größe aller Orakelanfragen.\\
    Wir müssen eine deterministische polynomiell zeitbeschränkte Turingmaschine $T'$ mit Orakel $A$ bauen, sodass $L^A(T') = L^A(T)$.

    Zunächst konstruieren wir eine deterministische Turingmaschine $T_{HILF}$, die ohne Orakelbenutzung die Sprache $L^A(T)$ entscheidet, indem alle Orakelanfragen ``mit Bordmitteln'' (also selbst) beantwortet werden unter Verwendung der Entscheidbarkeit von $A$.

    Vollständige Berechnungssequenzen einer Berechnung, deren Bandplatz $\leq k$ ist und $t$ Schritte lang ist, benötigen Platz $\bigO(k*t)$.\\
    Orakelanfragen einer Berechnung von $T$ auf $x$ (Eingabe) haben die Form $(S, y, 0^k)$ wobei $|S|, |y|, k \leq p(|x|)$.\\
    Wir verwenden also jetzt 3 Hilfsbänder, eines für die nichtdeterministische Berechnung von $T$ auf $x$, die wir der Reihe nach alle simulieren, eines für Orakelanfragen, eines für die Beantwortung der Orakelanfragen.\\
    Diese Maschine ist deterministisch und benötigt auf ihren Bändern höchstens $q(|x|)$ Platz, wobei $q$ ein von $p$ abgeleitetes Polynom ist (in etwa $q(n) = \bigO(p(n)^2$).\\

    Die eigentlichte Maschine $T'$ arbeitet jetzt wie folgt:\\
    Gegeben Eingabe $x$, schreibe $(T_{HILF}, x, O^{q(|x|)})$ auf das Orakelband. Falls $q_{yes}$, dann akzeptiere. Falls $q_{nein}$, dann verwerfe.

\end{beweis}


\begin{satz}
    
    $$ \exists B \in \Sigma^\ast: \ComplexityClassP^B \neq \ComplexityClassNP^B $$

\end{satz}

\begin{beweis}
    
    Falls $B \subseteq \Sigma^\ast$, 
    definiere $L_B = \{  0^k \ |\   \exists x \in \Sigma^\ast: |x| = k \land x \in B  \} $ \\
    Offensichtlich ist $L_B \in \ComplexityClassNP^B$, egal was $B$ ist.

    Es gilt jetzt, $B$ so zu wählen, dass für jede polynomiell zeitbeschränkte deterministische Orakel-Turingmaschine gitl: $L_B \neq L^B(T)$.

    Sei $i \mapsto T_i$ eine effektive Aufzählung von Orakel-Turingmaschinen, sodass $DTIME(T_i^x(x)) \leq |x|^i + i$ (alternativ $(|x| + 2)^i$ wie letztes Mal).\\
    Für alle deterministischen Orakel-Turingmaschinen $S$ und alle Orakel $x$ muss $i$ existieren, sodass $L(S^x) = L(T_i^x)$. $T_i$ ist die durch $i$ beschriebene Orakel-Turingmaschine künstlich auf Laufzeit $n^i + i$ beschränkt. Jetzt muss also für jedes $i$ ein $n_i$ existieren, sodass $T_i^B(0^{n_i})$ akzeptiert und $B$ enthält kein Wort der Länge $n_i$ (dann ist nämlich $0^{n_i} \notin L_B$), oder aber $T_i^B(0^{n_i})$ verwirft und $B$ enthält ein Wort $x_i$ mit $|x_i| = n_i$, denn dann ist $0^{n_i} \in L_B$.\\
    Dann ist in der Tat $L_B \notin \ComplexityClassP^B$.

    \textbf{Beobachtung:}
    Wenn $T_i^X(x)$ nach $t$ Schritten hält und $U$ aus Wörtern $y$ mit $|y| > t$ besteht, dann gilt 
    $T_i^{X \cup U}(x) \text{ akzeptiert }  \Leftrightarrow  \ T_i^X(x) \text{ akzeptiert}$.


\datum{19.11.15}

    
    Wir definieren rekursiv $n_i \in \N, B(i) \subseteq \Sigma^\ast, \Sigma = \{0,1\}, n_0 = 0, B(0) = \emptyset$
    \\
    Falls $B(0), \dots, B(i-1)$ schon definiert, definiere $n_i, B(i)$ so, dass gilt $\exists x: |x| = n_i \in B(i) \Leftrightarrow T_i^{B(i)}(0^{n_i} \text{ akzeptiert nicht}$.
    \\
    Außerdem sollte $T_i(0^{n_i})$ keine Elemente von $B(j) j>i$ anfragen.
    \\
    $B = \bigcup_{i \geq 0} B(i)$

    Es gilt dann $T_i^B(0^{n_i}) \text{ akzeptiert nicht } \Leftrightarrow T_i^{B(i)}(0^{n_i}) \text{ akzeptiert nicht } \Leftrightarrow 0^{n_i} \in L_{B(i)} \Leftrightarrow 0^{n_i} \in L_b$ (Falls wir zusätzlich dafür sorgen, dass $B(j)$ mit $j \geq i$ keine Elemente der Länge $n_i$ enthält.)

    Allgemein:
    $ B(i) =
    \begin{cases}
        B(i-1)              & , falls \dots     \\
        B(i-1) \cup \{x\}   & , falls \dots     \\
    \end{cases}
    $

    $n_i B(i) \in n_{i-1}(i-1)$
    $n_i$ sodass
    \begin{itemize}
        \item $n_i > n_{i-1}^{i-1} + i-1$ (Größer als die Laufzeit von $T_{i-1}(0^{n_{i-1}})$ und $T_j(0^{n_j})$ für $j < i$.)
        \item $2^{n_i} > n_i^i + i$ (da $2^x$ schneller wächst als $x^i+i$.)
    \end{itemize}
    Es gibt also mehr Wörter $x$ mit $|x| = n_i$ als die Laufzeit von $T_{i}(0^{n_i})$.

    Halbkonkrete Ausführung dieser Aufzählung für die ersten drei Schritte:
    \\
    $n_0 = 0, B(0) = \emptyset, n_1 > 0^0 + 0, 2^{n_1} > n_1 + 1$
    \\
    Rechne $T_1^{B(0)}(000)$. Wir nehmen an, dass nicht akzeptiert wird. Also $B(1) = B(0) = \emptyset$
    \\
    $n_2 > n_1^1 = 4, 2^{n_2} > n_2^2 + 2, \leadsto n_2 = 5$
    Rechne $T_2^{B(1)}(00000)$. Wir nehmen an, dass akzeptiert wird. Diese Rechnung dauerte $\leq 5^2 + 2 = 27$ Schritte. Es gibt 32 Wörter der Länge 5. Sei $x$ eines, das nicht abgefragt wurde. $x = 10110$. $B(2) = B(1) \cup \{10110\} = \{10110\}$
    \\
    $n_3 > 27, 2^{n_3} > n_3^3 + 3, \leadsto n_3 = 28$
    Rechne $T_3^{B(2)}(0^{28})$. Wir nehmen an, dass nicht akzeptiert wird. Also $B(1) = B(0) = \emptyset$
    \\
    \dots

    Invariante $B \cap \{ x \ |\  |x| \leq n_i \}  = B(i)$
    \\
    Rechenzeit von $T_i(0^{n_i}) < n_{i+1}$ ($> n_i^i + i$)

    $B$ unterscheidet sich von $B(i-1)$ nur durch Wörter die von $T_i^{B(i-1)}(0^{n_i})$ nicht angefragt werden. Entweder, da sie länger sind als die Rechenzeit oder von der Länge $n_i$ sind, aber so gewählt wurden, dass sie nicht gefragt werden.

    $T_i^B(0^{n_i}) \text{ akzeptiert } \\\Leftrightarrow 
    T_i^{B(i-1)}(0^{n_i}) \text{ akzeptiert } \\\Leftrightarrow 
    \neg \exists x \in B(i): |x| = n_i \\\Leftrightarrow 
    0^{n_i} \notin L_B
    $

    Das heißt $L(T_i^B) \neq L_B$

    $L_B \notin \ComplexityClassP^B$

\end{beweis}





\section{Polynomielle Hierarchie}

\begin{definition}
    
    $co\text{--}\phi = \{ L \subseteq \Sigma^\ast \ |\ \Sigma^\ast \setminus L \in \phi  \}$

\end{definition}

$co\text{--}\ComplexityClassP = \ComplexityClassP$

$co\text{--}\ComplexityClassNP$ ist nicht offensichtlich gleich $\ComplexityClassNP$ wegen der unsymmetrischen Akzeptanzbedingung bei Nichtdeterminismus.


\begin{definition}
    
    Die \definiere{polynielle Hierarchie} (PH):

    $\Delta_0^\ComplexityClassP = \ComplexityClassP$

    $\Sigma_0^\ComplexityClassP = \Pi_0^\ComplexityClassP = \ComplexityClassP$


    
    $\Sigma_1^\ComplexityClassP = \ComplexityClassNP$

    $\Pi^\ComplexityClassP = co\text{--}\ComplexityClassNP$

    Das soll andeuten, dass von der PH die Rede ist. Es hat nicht mit einem Exponenten oder Orakel zu tun.

\end{definition}


    

% \begin{beispiel}

%     $TAUT = \{ \psi \ |\  \psi \text{ allgemeingültig} \} \in co\text{--}\ComplexityClassNP$

%     $\Sigma^\ast \setminus TAUT = \{ \psi \ |\  \psi \text{ keine syntaktisch richtige Formel } \lor \} \in co\text{--}\ComplexityClassNP$

%     da war die tafel schon gewischt. naja.

% \end{beispiel}


\begin{definition}
    
    \definiere{relativierte Quantoren}


    Notation: $\exists_n x . A(x) \equiv \exists x \in \Sigma^\ast . |x| \leq n \land A(x)$

    Ebenso: $\forall_n x . A(x) \equiv \forall x \in \Sigma^\ast . |x| \leq n \Rightarrow A(x)$

    Außerdem: $\neg \exists_n x . A(x) \Leftrightarrow \forall_n x . \neg A(x)$

    und: $\neg \forall_n x . A(x) \Leftrightarrow \exists_n x . \neg A(x)$

\end{definition}

% Aus der Definition von $\ComplexityClassNP$ folgt

% man ich bin zu langsam und checken tu ichs auch nicht


\begin{definition}
    
    Die weitere \definiere{polynielle Hierarchie} (PH):

    $\Sigma_2^\ComplexityClassP = \ComplexityClassNP^{SAT} = \ComplexityClassNP^{\ComplexityClassNP}$

    $\Pi_2^\ComplexityClassP = co\text{--}\ComplexityClassNP^{SAT} = co\text{--}\ComplexityClassNP^{\ComplexityClassNP}$

    $\Delta_2^\ComplexityClassP = \ComplexityClassP^{SAT} = \ComplexityClassP^{\ComplexityClassNP}$

    $\Sigma_{n+1}^\ComplexityClassP = \ComplexityClassNP^{\Sigma_n^\ComplexityClassP}$

    $\Pi_{n+1}^\ComplexityClassP = co\text{--}\ComplexityClassNP^{\Sigma_n^\ComplexityClassP}$

    $\Delta_{n+1}^\ComplexityClassP = \ComplexityClassP^{\Sigma_n^\ComplexityClassP}$

    \textit{Anmerkung:}\\
    $\ComplexityClassNP^\phi = \bigcup_{X \in \phi} \ComplexityClassNP^X$


\end{definition}





\section{Kollabieren der polynomiellen Hierarchie}



Falls $\ComplexityClassP = \ComplexityClassNP$, so folgt $\Sigma_n^\ComplexityClassP = \ComplexityClassP$.

Im allgemeinen gilt: Falls $\Sigma_{n+1}^\ComplexityClassP = \Sigma_n^\ComplexityClassP$ für ein bestimmtes $n$, dann auch $\Sigma_{n'}^\ComplexityClassP = \Sigma_n^\ComplexityClassP$ für alle $n' \geq n$. Man sagt dann PH kollabiere auf der $n$. Stufe.





\datum{23.11.15}


\begin{satz}
    
    $L \in \Sigma_2^\ComplexityClassP \Leftrightarrow \\
    \exists L' \in \ComplexityClassP, \text{ Polynom } p:
    x \in L \Leftrightarrow \exists_{p(|x|)} y . \forall_{p(|x|)} z . (x,y,z) \in L'$

\end{satz}

\begin{beweis}
    
    \begin{itemize}

        \item ``$\Leftarrow$'':

            Seien $L' \in \ComplexityClassP$, $p$ Polynom vorgegeben.\\

            Gesucht: nichtdeterministische Polynomialzeit-Orakel-Turingmaschine $M$ sodass $M$ mit geeignetem $\ComplexityClassNP$-Orakel die Sprache $L$ entscheide.

            $L'' = \{ (x,y) \ |\  \exists_{p(|x|)} z . (x,y,z) \notin L' \} $

            $L'' \in \ComplexityClassNP$ (Rate $z$ und prüfe $(x,y,z) \notin L'$)

            $x \in L \Leftrightarrow \exists_{p(|x|)} y . (x,y) \notin L'' $

            Die Maschine $M$ rät also $y$ und prüft $(x,y) \notin L''$ mit Orakel für $L''$.

        \item ``$\Rightarrow$'':

            Sei $M$ eine nichtdeterministische durch $p(n)$ laufzeitbeschränkte Turingmaschine mit Orakel $X \in \ComplexityClassNP$, z.B. $X = SAT$, $L = L(M)$

            Es ist $x \in L = L(M)$ genau dann, wenn \\
            $\exists \text{ Lauf } y \text{ von } M \text{ auf }: 
            M \text{ akzeptiert } x \land |y| \leq q(|x|)$ \\
            wobei $q$ ein Polynom ist mit $q(n) = \bigO(p(n)^2)$

            Um zu prüfen, ob $y$ tatsächlich ein LAuf ist und noch dazu akzeptierend, muss neben allem möglichen, was in polynomieller Zeit geht, z.B. 
            \begin{itemize}
                \item Folgefkonfigurationen jeweils gemäß der Maschinentafel ($\delta_M$) aus Vorgängerkonfigurationen enthalten,
                \item Am Anfang Startzustand, am Ende akzeptierender Zustand,
                \item Input okay in Startkonfiguration kopiert,
            \end{itemize}
            auch geprüft werden, dass alle Orakelanfragen richtig beantwortet wurden.

            \dots 

            und

            $\exists \eta . \eta_i$ erfüllt die $i.$ \underline{positiv} beantwortete Orakelanfrage $q_i \in SAT$.

            und

            $\forall \rho . \rho_j$ erfüllt die $j.$ \underline{negativ} beantwortete Orakelanfrage $\rho_j \in SAT$ \underline{nicht}.

            Es gibt also ein Polynom $q(n)$ sodass $x \in L(M) \Leftrightarrow \exists_{q(|x|)} y . L_1(x,y) \land \exists_{p(|x|)} \eta . L_2(y,\eta) \land \forall_{q(|x|)} \rho . L_3(y, \rho)$ \\
            wobei $L_1, L_2, L_3 \in \ComplexityClassP$

            \begin{itemize}
                \item 
                    $L_1(x,y) \Longleftrightarrow y \text{ ist akzeptierender Lauf von } M \text{ auf } x \text{ bis auf Orakelanfragen}$

                \item 
                    $L_2(y,\eta) \Longleftrightarrow  \text{ Die in } y \text{ positiv beantworteten Orakelanfragen sind } \rho_1 \dots \rho_k, \eta =| y_1 \dots y_k \text{ und } \eta_i \vDash \psi_i \text{ für } i = 1 \dots k$

                \item 
                    $L_3(y, \rho) \Longleftrightarrow \text{ Die in } y \text{ negativ beantworteten Orakelanfragen sind } \psi_1 \dots \psi_k, \rho = \rho_1 \dots \rho_k \text{ und } \rho \mp \psi_i \text{ für } i = 1 \dots k$

            \end{itemize}

    \end{itemize}

\end{beweis}



\begin{korollar}
    
    $L \in \Pi_2^\ComplexityClassP \Longleftrightarrow \\
    \exists \text{ Polynom } p \land L' \in \ComplexityClassP: x \in \Leftrightarrow \forall_{p(|x|)} y . \exists_{p(|x|)} z . (x,y,z) \in L'$ 

\end{korollar}


\begin{satz}
    
    Ist $L \in \Sigma_n^\ComplexityClassP$, so existiert ein Polynom $p(n)$ und $L' \in \ComplexityClassP$ sodass\\
     $x \in L \Leftrightarrow y_1 \forall_{p(|x|)} y_2 \exists_{p(|x|)} y_3 \dots   Q_{p(|x|)} y_n . (x, y_1, \dots y_n) \in L'$\\
    $n$ gerade: $Q = \forall$\\
    $n$ ungerade: $Q = \exists$

\end{satz}

Beweis durch Induktion über $n$.











\begin{satz}

    \definiere{Satz von Kamp-Lipton}

    Falls $SAT$ polynomiell große Schaltkreise hat, so kollabiert die PH auf der zweiten Stufe.\\
    Das heißt $\forall n \geq 2 : \Sigma_n^\ComplexityClassP = \Sigma_2^\ComplexityClassP$.


    Dass $SAT$ ``polynomielle Schaltkreise hat'' soll heißen: Es gibt ein Polynom $p(n)$ und für jedes $n \in \N$ einen boolschen Schaltkreis $C_n$ mit $n$ Inputs und Größe $\leq p(n)$ ($|C_n| \leq p(n)$).\\
    Für alle aussagenlogische Formeln $\phi$ mit $|\phi| = n$ gilt $\phi \in SAT \Leftrightarrow C(\phi) = TRUE$, $C(\phi)$ die Bitkodierung von $\phi$ an die $n$ Inputs von $C$ anlegen.
    
\end{satz}

Kurznotation: $SAT \in \ComplexityClassP / poly$

Der Satz von Kamp-Lipton sagt also $SAT \in \ComplexityClassP / poly \Rightarrow \text{ PH } = \Sigma_2^\ComplexityClassP$

Hilfsmittel: ``Selbstreduzierbarkeit von SAT''\\
Ein Schaltkreis $C_n$ wie oben beschrieben kann so umgebaut werden in einem Schaltkreis $D_n$ polynomielle Größe, dass bei Antwort TRUE eine erfüllende Belegung zurückgeliefert wird.

Das heißt $D_n$ hat $n$ Ausgänge, die eine Belegung kodieren sollen. Spezifiere 
\begin{itemize}
    \item 
        $\phi \in SAT . D_n(\phi) (\eta, TRUE)$ mit $\eta \vDash \phi$
    \item 
        $\phi \notin SAT . D_n(\phi) (\_, FALSE)$
\end{itemize}
$D_n$ ruft $C_n$ insgesamt $m \leq n$ mal auf, wobei $m$ die Zahl der Variablen plus eins ist.

% Wir zeigen jetzt unter Zuhilfenahme von $(D_n)_n$, dass $\Pi_2^\ComplexityClassP = \Sigma_2^\ComplexityClassP $:
% \begin{itemize}

%     \item ``$\subseteq$'':

%         Sei $L \in \Pi_2^\ComplexityClassP_2$ und (nach Satz) $\forall_{p(|x|)} y . \exists_{p(|x|)} z . (x,y,z) \in L'$ ($p$ Polynom, $L' \in \ComplexityClassP$)

%         $x \in L \Longleftrightarrow \exists_{q(|x|)}  D . D \text{ ist selbstverifizierender SAT-Tester so wie } (D_n)_n, $ \\
%         und\\
%         $\forall_{p(|x|)} y . D(``\exists_{p(|x|)} z . (x,y,z) \in L' '')  =  (\_, TRUE)$\\
%         $\Leftrightarrow$\\
%         % $\forall_{poly} \phi \forall_{poly} \eta . D(\phi) = (\_, FALSE) \land \eta \mp \phi \\
%         % \lor $\
%         blablabla



% \end{itemize}





% chapter np_und_p (end)

% %!TEX root = 0-main.tex

% Author: Philipp Moers <soziflip@gmail.com> 



\chapter{Platzkomplexität} % (fold)
\label{cha:platzkomplexitaet}
    



\section{Platzkonstruierbare Funktionen}


\begin{definition}
    
    Eine Funktion $s(n)$ heißt \definiere{platzkonstruierbar} genau dann, wenn eine deterministische Turingmaschine existiert, die bei Eingabe $0^n$ genau $s(n)$ Bandfelder beschreibt und dann hält.

\end{definition}

\begin{beispiel}

    Alle Polynome mit Koeffizienten $\in \Q^+$, die Wurzelfunktion, die Logarithmus-Funktion, die Potzenfunktion usw. sind platzkonstruierbar.
    
\end{beispiel}




\section{Platzverbrauch einer Turingmaschine}

\begin{definition}

    Der \definiere{Platzverbrauch einer Turingmaschine} (deterministisch oder nichtdeterministisch) bei Eingabe $x$ ist 

    \begin{itemize}

        \item \textbf{erste Definition}\\ 
        die Größe des beschriebenen Teils aller Bänder am Ende der Berechnung. (Mit dieser Definition ist der Platzverbrauch stets $\geq |x|$).

        \item \textbf{zweite Definition}\\ 
        die Endgröße aller anderen Bänder, wobei das Eingabeband nicht überschrieben werden darf.

    \end{itemize}

    Die zweite Definition ist Standard,w enn sublineare Platzschranken betrachtet werden, zum Beispiel $\log(n)$. Oberhalb von $\bigO(n)$ sind die beiden Definitionen äquivalent.
    


    Notation: $DSPACE_M (x)$ und $NSPACE_M (x)$

    $DSPACE_M (s(n)) = \{   L  \ |\   \exists DTM M : L = L(M) \land DSPACE_M(x) = \bigO(s(|x|)) \}$

    $NSPACE_M (s(n)) = \{   L  \ |\   \exists DTM M : L = L(M) \land NSPACE_M(x) = \bigO(s(|x|)) \}$

    $PSPACE = \bigcup_{k \geq 0} DSPACE(n^k) $ (polynomieller Platz)

    $LINSPACE = DSPACE(n)$

    $LOGSPACE = DSPACE(\log n)$ (auch als $L$ bezeichnet)

    $NLOGSPACE = NSPACE(\log n)$ (auch als $NL$ bezeichnet)


\end{definition}


\begin{beispiel}
    
    STCONN (Erreichbarkeit in gerichteten Graphen) ist $\in NLOGSPACE$ (rate Pfad) und $\in LINSPACE$ (Tiefensuche/Breitensuche)

\end{beispiel}


Es gibt eine triviale, aber wissenswerte Beziehung zwischen Zeit- und Platzkomplexität:

% $$ NSPACE(s(n)) \subseteq DTIME (2^{\bigO(s(n))}) $$      % not sure
$$ DSPACE(s(n)) \subseteq DTIME (2^{\bigO(s(n))}) $$

% Alle Berechnungsfolgen können aufgezählt werden.
Hat die Berechnung nach $2^{c*s(n)}$ Schritten nicht geendet, so kann abgebrochen werden wegen Wiederholung einer globalen Konfiguration. (Tatsächlicher Platzverbrauch $\leq c * s(n)$)



$$ DTIME(t(n)) \subseteq DSPACE (t(n)) $$

Mehr Platz als Laufzeit kann nicht angefordert werden.





\begin{satz}

    Für deterministische Einband-Turingmaschinen $T$ gilt:

    $DTIME_T(x) = \bigO(t(|x|) \Longrightarrow  L(T) \in DSPACE(\sqrt{t(n)})$

    
\end{satz}

Für Mehrband-Turingmaschinen gibt es einen ähnliches Satz, bei dem der Platz allerdings etwas größer ist. Dass er für Einband-Turingmaschinen so gut ist, ist gewissermaßen kurios.





\section{Platzhierarchiesatz}

\begin{satz}
    
    ``Echt mehr Platz hilft auch mehr.''

    \textit{Für genaue Aussage und Beweis siehe z.B. Papadimitrion}
\end{satz}

Wichtige Konsequenz:

$$ LOGSPACE \subset PSPACE $$

$$ LOGSPACE \subseteq NLOGSPACE \subseteq \ComplexityClassP \subseteq \ComplexityClassNP \subseteq PH \subseteq PSPACE $$

Von jeder dieser Inklusionen ist unbekannt, ob sie echt sind. Mindestens eine muss aber echt sein.



\begin{satz}
    
    \definiere{Satz von Savitch}

    Für eine platzkonstruierbare Funktion $s(n) \geq \log(n)$ ist \\
    $ NSPACE(s(n)) \subseteq DSPACE(s(n)^2)$

    (Vergleiche
    $ NTIME(t(n)) \subseteq DTIME(2^{\bigO(t(n))})$
    )

\end{satz}

\begin{beweis}
    
    Sei eine nichtdeterministische Turingmaschine $T$ gegeben mit Platzbedarf $S = c * s(|x|)$ bei Eingabe $x$. Wir betrachten eine Kodierung der globalen Konfigurationen von $T(x)$ durch Wörter der Länge $S$ und o.B.d.A gebe es exakt eine akzeptierende Endkonfiguration $s_{ACC}$. (Alle Bänder am Ende löschen, d.h. mit $0$ überschreiben.)

    $
    x \in L(T) 
        \Longleftrightarrow 
    s_{ACC} \text{ von } s_{INI} \text{ aus in } \leq 2^S \text{ Schritten erreichbar}
    $

    Hier steht $s_{INI}$ für die Startkonfiguration bei Eingabe $x$. $2^S$ ist die Gesamtzahl der Konfigurationen.

    Das heißt $s_{ACC}$ ist von $s_{INI}$ aus im Graphen der Konfigurationen erreichbar (Spezialfall von STCONN).

\end{beweis}


\textbf{Notation:}

$s \rightarrow_T s'$: $s'$ ist 1-Schritt-Folgekonfiguration von $s$ in $T$ und kann in $LOGSPACE$ entschieden werden.

$REACH(s, s')$: $s \rightarrow^\ast s'$ ($s'$ ist von $s$ erreichbar)

$REACH(s, s', i)$: $s \rightarrow^{\leq 2^i} s'$ ($s'$ ist von $s$ in weniger als $2^i$ Schritten erreichbar)


$ x \in L(T) \Longleftrightarrow REACH(s_{INI}, s_{ACC}, S)$

Es gilt 
$$REACH(s, s', 0) \Longleftrightarrow s = s' \lor s \rightarrow_T s'$$
\hspace{4cm}($2^0 = 1$)
$$REACH(s, s', i+1) \Longleftrightarrow \exists \check{s} : REACH(s, \check{s}, i) \land REACH(\check{s}, s', i)$$ 
\hspace{4cm}($2^{i+1} = 2 * 2^i$)\\
Dies liefert eine rekursive Implementierung von $REACH(s,s',i)$ \\
\hspace{4cm}($\exists \check{s} \leadsto \text { for } \check{s} \in \text{ globale Konfigurationen}$)


Der Rekursionsstack hat Tiefe $S$ (Toplevel-Aufruf $REACH(s_{INI}, s_{ACC}, S)$).
\\
Jeder Activationrecord hat Größe $\bigO(S)$ genauer gesagt $2 S$ für die beiden Parameter $s, s', \log(S)$ für die Parameter $i$. Wenn gewünscht noch ein weiteres $S$ für die for-Schleife.


Die Gesamtgröße des Stacks ist beschränkt durch $S * \bigO(S) = \bigO(S^2)$.


Historisch wurde zunächst gezeigt, dass STCONN in $DSPACE(\log(n)^2)$ liegt. Der Satz von Savitch kann auch hieraus abgeleitet werden.






% chapter platzkomplexitaet (end)

% \input{5-alternierung-und-hierarchien.tex}
% \input{6-schaltkreise.tex}
% \input{7-interaktive-und-probabilistische-algorithmen.tex}


\end{document}
